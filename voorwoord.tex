%%=============================================================================
%% Voorwoord
%%=============================================================================

\chapter*{Voorwoord}
\label{ch:voorwoord}

%% TODO:
%% Het voorwoord is het enige deel van de bachelorproef waar je vanuit je
%% eigen standpunt (``ik-vorm'') mag schrijven. Je kan hier bv. motiveren
%% waarom jij het onderwerp wil bespreken.
%% Vergeet ook niet te bedanken wie je geholpen/gesteund/... heeft

Deze bachelorproef duidt het einde aan van mijn opleiding Toegepaste Informatica. Met deze bachelorproef wil ik bewijzen dat ik op een zelfstandige en objectieve manier onderzoek kan voeren over een (nieuwe) technologie in de IT-wereld. Omdat onze branche vrijwel continu onderhevig is aan verandering, vind ik dit een niet-onbelangrijke competentie. 

Als onderwerp van deze bachelorproef heb ik besloten om een al wat oudere (maar zeker niet oninteressante) technologie te bespreken: het ZFS bestandssysteem. De redenen waarom ik net dit onderwerp zou willen bespreken, zijn nogal uiteenlopend. Ik ben zelf een enorme Linux- en Unix-fan en ik hou ervan om mezelf nieuwe dingen aan te leren. Met ZFS was ik nog niet vertrouwd, en daar ik toch van plan ben om zelf een homeserver samen te stellen en als bestandssysteem ZFS te gebruiken, leek deze bachelorproef mij een uitstekende opportuniteit om mij wat meer te verdiepen in de werking en implementatie van deze technoogie. Ook hoorde ik hier en daar geruchten vallen over de mogelijke onbetrouwbaarheid van RAID (het zgn.RAID 5 "write hole") en dit was dan ook een reden om onderzoek te voeren naar een mogelijk alternatief.

Deze bachelorproef is het resultaat van vele uren noeste arbeid; het is een werk waar ik enorm trots op ben. Desalniettemin zou dit werk niet mogelijk zijn geweest zonder de hulp van een aantal mensen. Graag neem ik daarom even de tijd om enkele personen te bedanken voor hun steun en toeverlaat gedurende deze periode. 

Eerst en vooral zou ik mijn promotor, mevr. Antonia Pierreux, en mijn co-promotor, mevr. Karine Van Driessche, willen bedanken voor het goed laten verlopen van deze periode. Het schrijven van deze scriptie en het voeren van een onderzoek waren geen makkelijke karweien; zonder hun inhoudelijke en technische steun zou deze bachelorproef nooit mogelijk geweest zijn. Tevens zou ik ook nog graag mijn familie en vrienden willen bedanken voor de onophoudelijke steun en begrip. Ook wil ik graag de mensen van DViT bedanken voor het aanreiken van technische kennis en materiaal voor mijn onderzoek. En \textit{last but not least} wil ik mijn ouders enorm bedanken om mij gedurende deze drie jaar ten volle te steunen: dankzij hen heb ik telkens opnieuw de moed teruggevonden om er met volle teugen tegenaan te gaan in perioden dat het wat moeilijker ging.

Ik hoop dat u evenveel plezier beleeft met het lezen van mijn scriptie als ik had met het schrijven ervan.

\begin{flushright}
  \textit{Jonas De Moor}
  \textit{Academiejaar 2016-2017}
\end{flushright}
