%%=============================================================================
%% LaTeX sjabloon voor bachelorproef, HoGent Bedrijf en Organisatie
%% Opleiding Toegepaste Informatica
%%=============================================================================

\documentclass[fleqn,a4paper,12pt]{book}

\input{structure}

%%---------- Documenteigenschappen --------------------------------------------
%% TODO: Vul dit aan met je eigen info:

% Je eigen naam
\newcommand{\student}{Jonas De Moor}

% De naam van je promotor (lector van de opleiding)
\newcommand{\promotor}{Antonia Pierreux}

% De naam van je co-promotor. Als je promotor ook je opdrachtgever is en je
% dus ook inhoudelijk begeleidt (en enkel dan!), mag je dit leeg laten.
\newcommand{\copromotor}{Karine Van Driessche}

% Indien je bachelorproef in opdracht van/in samenwerking met een bedrijf of
% externe organisatie geschreven is, geef je hier de naam. Zoniet laat je dit
% zoals het is.
\newcommand{\instelling}{---}

% De titel van het rapport/bachelorproef
\newcommand{\titel}{ZFS met RAID-Z als alternatief voor klassieke RAID-oplossingen}

% Datum van indienen (gebruik telkens de deadline, ook al geef je eerder af)
\newcommand{\datum}{2 juni 2017}

% Academiejaar
\newcommand{\academiejaar}{2016-2017}

% Examenperiode
%  - 1e semester = 1e examenperiode => 1
%  - 2e semester = 2e examenperiode => 2
%  - tweede zit  = 3e examenperiode => 3
\newcommand{\examenperiode}{2}


% Woordenlijst

% Andere naam voor woordenlijst
%\renewcommand{\glossaryname}{Gebruikte termen}

\newglossaryentry{performantie}
{
  name={Performantie},
  description={Performantie (Engels: performance) is een vrij uitgebreid begrip in de informatica. In het algemeen bedoelt men met performantie meestal hoe goed een computersysteem vooropgestelde taken kan uitvoeren in een bepaalde situatie. Bij het meten van performantie van een systeem kan men één of meerdere aspecten beschouwen: bij schijven kan bijvoorbeeld het aantal I/O's (Invoer- en uitvoerbewerkingen) worden genomen als maatstaf; bij CPU's kan een maatstaf bijvoorbeeld het aantal instructies per seconde zijn dat deze kan verwerken}
}

\newglossaryentry{betrouwbaarheid}
{
  name={Betrouwbaarheid},
  description={Betrouwbaarheid (Engels: reliability) heeft betrekking op de betrouwbaarheid van een computersysteem, i.e. of deze al dan niet zijn taken op een juiste manier uitvoert. Men kan bijvoorbeeld stellen dat er zich bij een computersysteem een X aantal fouten mogen voordoen; als deze grens overschreden wordt, dan wordt het systeem als onbetrouwbaar verklaard. Daarnaast heeft betrouwbaarheid ook te maken met de handelswijze van een systeem indien er zich een fout voordoet: men kan zich bijvoorbeeld afvragen hoe er zou moeten gereageerd worden op een kritieke fout en hiervoor test cases opstellen om de betrouwbaarheid van het systeem te testen}
}

\newglossaryentry{capaciteit}
{
  name={Capaciteit},
  description={In de context van schijven of RAID-systemen kan capaciteit (Engels: capacity) een aantal betekenissen hebben. In deze scriptie wordt er vooral gesproken over de bruikbare capaciteit van een schijf of RAID-array: dit is de hoeveelheid schijfruimte die kan gebruikt worden door gebruikers om data op te slaan. Daarnaast kan men ook spreken over de totale capaciteit: dit is de som van de gebruikte schijfruimte en de vrije schijfruimte. In theorie is de totale capaciteit van een schijf gelijk aan de werkelijke grootte van een schijf. }
}

\newglossaryentry{atomair}
{
  name={Atomair},
  description={In de informatica betekent atomiciteit (Engels: atomicity) hetzelfde als ondeelbaarheid; vaak wordt dit begrip ook omschreven als een "alles-of-niets" aanpak. In de context van bijvoorbeeld databanksystemen heeft atomiciteit betrekking op o.a. transacties: ofwel verloopt een transactie succesvol, ofwel mislukt deze (door bv. een onderbreking) en moet de databank kunnen teruggebracht worden naar een consistente toestand. Meestal betekent dit dan dat er een rollback moet worden uitgevoerd.} 
}


\makenoidxglossaries

%%=============================================================================
%% Inhoud document
%%=============================================================================

\begin{document}

%---------- Taalselectie ------------------------------------------------------
%% Als je je bachelorproef in het Engels schrijft, haal dan onderstaande regel
%% uit commentaar. Let op: de tekst op de voorkaft blijft in het Nederlands, en
%% dat is ook de bedoeling!
%\selectlanguage{english}

%---------- Titelblad ---------------------------------------------------------
\inserttitlepage

%---------- Samenvatting, voorwoord -------------------------------------------
\usechapterimagefalse
%%=============================================================================
%% Samenvatting
%%=============================================================================

%% TODO: De "abstract" of samenvatting is een kernachtige (~ 1 blz. voor een
%% thesis) synthese van het document.
%%
%% Deze aspecten moeten zeker aan bod komen:
%% - Context: waarom is dit werk belangrijk?
%% - Nood: waarom moest dit onderzocht worden?
%% - Taak: wat heb je precies gedaan?
%% - Object: wat staat in dit document geschreven?
%% - Resultaat: wat was het resultaat?
%% - Conclusie: wat is/zijn de belangrijkste conclusie(s)?
%% - Perspectief: blijven er nog vragen open die in de toekomst nog kunnen
%%    onderzocht worden? Wat is een mogelijk vervolg voor jouw onderzoek?
%%
%% LET OP! Een samenvatting is GEEN voorwoord!

%%---------- Nederlandse samenvatting -----------------------------------------
%%
%% TODO: Als je je bachelorproef in het Engels schrijft, moet je eerst een
%% Nederlandse samenvatting invoegen. Haal daarvoor onderstaande code uit
%% commentaar.
%% Wie zijn bachelorproef in het Nederlands schrijft, kan dit negeren en heel
%% deze sectie verwijderen.

%\IfLanguageName{english}{%
%\selectlanguage{dutch}
%\chapter*{Samenvatting}
%\lipsum[1-4]
%\selectlanguage{english}
%}{}

%%---------- Samenvatting -----------------------------------------------------
%%
%% De samenvatting in de hoofdtaal van het document

\chapter*{\IfLanguageName{dutch}{Samenvatting}{Abstract}}

%\lipsum[1-4]

RAID (Redundant Array of Independent Disks) is een technologie die al lange tijd is ingeburgerd in bedrijven. Systeembeheerders gebruiken RAID vooral om ervoor te zorgen dat één of meerdere defecte schijven niet kunnen leiden tot dataverlies, maar dit hoeft niet noodzakelijk het geval te zijn. 

In het laatste decennium zijn softwaregebaseerde RAID-oplossingen steeds populairder geworden. Eén van deze oplossingen is RAID-Z, een softwarematige RAID die deel uitmaakt van de ZFS storage stack. ZFS is een geavanceerd bestandssysteem ontwikkeld door het vroegere Sun Microsystems in het begin van de jaren 2000. 

In deze bachelorproef wordt achterhaald of ZFS en RAID-Z een goed alternatief zouden vormen voor meer traditionele RAID-oplossingen, zoals een hardwaregebaseerde RAID. De motivatie voor het voeren van een onderzoek naar een alternatieven voor RAID is hoofdzakelijk omdat traditionele RAID5 arrays nog steeds onderhevig zijn aan het zgn. "RAID5 write hole", waarbij dataverlies kan optreden bij bijvoorbeeld een stroompanne. Ook bieden de meeste RAID-controllers geen bescherming tegen silent data corruption.

De scriptie is opgedeeld in twee grote delen: een theoretisch deel en een praktisch deel. In het theoretische deel worden de basisprincipes van RAID en de architectuur van ZFS in vogelvlucht overlopen. Het praktische gedeelte behandelt voornamelijk de performantie en betrouwbaarheid van ZFS en RAID-Z. 
De performantietesten en het grootste deel van het praktische deel werd uitgevoerd m.b.v. een HP desktopsysteem; de betrouwbaarheidstesten werden uitgevoerd met een virtuele machine via VirtualBox. Hierbij werd geconstateerd dat ZFS zijn beloftes m.b.t. betrouwbaarheid waarmaakt en de data goed beschermt tegen datacorruptie en hardwarefalen. Performantie van ZFS is uitstekend te noemen; de performantie werd vergeleken met die van Linux MD, een softwarematige RAID voor Linux, en werd getest m.b.v. de Phoronix Test Suite.

Afhankelijk van de use case (situatie, nodige opslagcapaciteit, beschikbare hardware) vormt ZFS met RAID-Z in het merendeel van de gevallen een uitstekend alternatief voor een klassieke RAID-oplossing. In de toekomst zullen ZFS en andere COW-bestandssystemen, zoals APFS, ReFS en BTRFS, naar alle waarschijnlijkheid interessanter worden voor dagelijks gebruik; een vergelijkende studie tussen deze verschillende bestandssystemen zou nog een interessante toevoeging zijn aan deze scriptie.

%%=============================================================================
%% Voorwoord
%%=============================================================================

\chapter*{Voorwoord}
\label{ch:voorwoord}

%% TODO:
%% Het voorwoord is het enige deel van de bachelorproef waar je vanuit je
%% eigen standpunt (``ik-vorm'') mag schrijven. Je kan hier bv. motiveren
%% waarom jij het onderwerp wil bespreken.
%% Vergeet ook niet te bedanken wie je geholpen/gesteund/... heeft

Deze bachelorproef duidt het einde aan van mijn opleiding Toegepaste Informatica. Met deze bachelorproef wil ik bewijzen dat ik op een zelfstandige en objectieve manier onderzoek kan voeren over een (nieuwe) technologie in de IT-wereld. Omdat onze branche vrijwel continu onderhevig is aan verandering, vind ik dit een niet-onbelangrijke competentie. 

Als onderwerp van deze bachelorproef heb ik besloten om een al wat oudere (maar zeker niet oninteressante) technologie te bespreken: het ZFS bestandssysteem. De redenen waarom ik net dit onderwerp zou willen bespreken, zijn nogal uiteenlopend. Ik ben zelf een enorme Linux- en UNIX-fan en ik hou ervan om mezelf nieuwe dingen aan te leren. Met ZFS was ik nog niet vertrouwd, en daar ik toch van plan ben om zelf een homeserver samen te stellen en als bestandssysteem ZFS te gebruiken, leek deze bachelorproef mij een uitstekende opportuniteit om mij wat meer te verdiepen in de werking en implementatie van deze technologie. Ook hoorde ik hier en daar geruchten vallen over de mogelijke onbetrouwbaarheid van RAID (het zgn.RAID 5 "write hole") en dit was dan ook een reden om onderzoek te voeren naar een mogelijk alternatief.

Deze bachelorproef is het resultaat van vele uren noeste arbeid; het is een werk waar ik enorm trots op ben. Desalniettemin zou dit werk niet mogelijk zijn geweest zonder de hulp van een aantal mensen. Graag neem ik daarom even de tijd om enkele personen te bedanken voor hun steun en toeverlaat gedurende deze periode. 

Eerst en vooral zou ik mijn promotor, mevr. Antonia Pierreux, en mijn co-promotor, mevr. Karine Van Driessche, willen bedanken voor het goed laten verlopen van deze periode. Het schrijven van deze scriptie en het voeren van een onderzoek waren geen makkelijke karweien; zonder hun inhoudelijke en technische steun zou deze bachelorproef nooit mogelijk geweest zijn. Tevens zou ik ook nog graag mijn familie en vrienden willen bedanken voor de onophoudelijke steun en begrip. Ook wil ik graag de mensen van DViT bedanken voor het aanreiken van technische kennis en materiaal voor mijn onderzoek. En \textit{last but not least} wil ik mijn ouders enorm bedanken om mij gedurende deze drie jaar ten volle te steunen: dankzij hen heb ik telkens opnieuw de moed teruggevonden om er met volle teugen tegenaan te gaan in perioden dat het wat moeilijker ging.

Ik hoop dat u evenveel plezier beleeft met het lezen van mijn scriptie als ik had met het schrijven ervan.

\begin{flushright}
  \textit{Jonas De Moor}
  \textit{Academiejaar 2016-2017}
\end{flushright}


%---------- Inhoudstafel ------------------------------------------------------
\pagestyle{empty} % No headers
\tableofcontents % Print the table of contents itself
\cleardoublepage % Forces the first chapter to start on an odd page so it's on the right
\pagestyle{fancy} % Print headers again

%---------- Lijst afkortingen, termen -----------------------------------------
%% Als je een lijst van afkortingen of termen wil toevoegen, dan hoort die
%% hier thuis. Gebruik bijvoorbeeld de ``glossaries'' package.



%%---------- Kern -------------------------------------------------------------

%%=============================================================================
%% Inleiding
%%=============================================================================

\chapter{Inleiding}
\label{ch:inleiding}

%De inleiding moet de lezer alle nodige informatie verschaffen om het onderwerp te begrijpen zonder nog externe werken te moeten raadplegen \autocite{Pollefliet2011}. Dit is een doorlopende tekst die gebaseerd is op al wat je over het onderwerp gelezen hebt (literatuuronderzoek).

%Je verwijst bij elke bewering die je doet, vakterm die je introduceert, enz. naar je bronnen. In \LaTeX{} kan dat met het commando \texttt{$\backslash${textcite\{\}}} of \texttt{$\backslash${autocite\{\}}}. Als argument van het commando geef je de ``sleutel'' van een ``record'' in een bibliografische databank in het Bib\TeX{}-formaat (een tekstbestand). Als je expliciet naar de auteur verwijst in de zin, gebruik je \texttt{$\backslash${}textcite\{\}}.
%Soms wil je de auteur niet expliciet vernoemen, dan gebruik je \texttt{$\backslash${}autocite\{\}}. Hieronder een voorbeeld van elk.

%\textcite{Knuth1998} schreef een van de standaardwerken over sorteer- en zoekalgoritmen. Experten zijn het erover eens dat cloud computing een interessante opportuniteit vormen, zowel voor gebruikers als voor dienstverleners op vlak van informatietechnologie~\autocite{Creeger2009}.

%\section{Stand van zaken}
%\label{sec:stand-van-zaken}

%% TODO: deze sectie (die je kan opsplitsen in verschillende secties) bevat je
%% literatuurstudie. Vergeet niet telkens je bronnen te vermelden!

%\lipsum[7-20]

In dit hoofdstuk wordt er een korte inleiding gegeven over de basisprincipes van RAID, aangezien RAID-Z een softwarematige vorm van RAID is. Daarnaast wordt de geschiedenis en globale werking van ZFS reeds kort besproken. Aan het einde van dit hoofdstuk kunnen de probleemstelling en onderzoeksvragen worden teruggevonden, samen met de verdere indeling van deze bachelorproef.

\section{RAID}

Al van oudscher worden magnetische harde schijven gebruikt als opslagmedium voor data. Maar reeds in de jaren 80 zagen onderzoekers in dat I/O-performance een bottleneck zou vormen voor computersystemen in de toekomst. Terwijl geheugenchips en processoren steeds sneller werden, bevonden opslagmedia zich in een impasse \autocite{DavidA.Paterson1987}.

In de paper \textit{"A Case for Redundant Arrays of Inexpensive Disks"}  formuleerden \textcite{DavidA.Paterson1987} en zijn collega's voor het eerst de term 'RAID', wat een acroniem is voor 'Redundant Arrays of Inexpensive Disks'. De oorspronkelijke idee achter RAID was dat een verzameling van goedkopere schijven performanter zou zijn dan grotere en duurdere mainframeschijven van die tijd. 

Naast performantie en kost was betrouwbaarheid (reliability)  ook een belangrijke factor. Als bv. één of meerdere schijven van de array falen, dan mag dit geen invloed hebben op de werking van de rest van de verzameling schijven. Daarom introduceerden de onderzoekers de zgn. "RAID levels"    \autocite{DavidA.Paterson1987}, die vandaag de dag nog steeds in gebruik zijn. Er bestaan een aantal RAID-niveaus, waarvan de voornaamste zullen besproken worden.

Bij het bouwen van RAID-systemen worden er gebruikelijk drie aspecten in beschouwing genomen: \textbf{performantie} (performance), \textbf{betrouwbaarheid} (reliability) en \textbf{capaciteit} (capacity). Een RAID-niveau is in principe niets anders dan een balans tussen deze verschillende eigenschappen; meestal zullen er dus één of meerdere trade-offs moeten gemaakt worden \autocite{OSThreePiecesRemzi2015}.

\section{Probleemstelling en Onderzoeksvragen}
\label{sec:onderzoeksvragen}

%% TODO:
%% Uit je probleemstelling moet duidelijk zijn dat je onderzoek een meerwaarde
%% heeft voor een concrete doelgroep (bv. een bedrijf).
%%
%% Wees zo concreet mogelijk bij het formuleren van je
%% onderzoeksvra(a)g(en). Een onderzoeksvraag is trouwens iets waar nog
%% niemand op dit moment een antwoord heeft (voor zover je kan nagaan).

\section{Opzet van deze bachelorproef}
\label{sec:opzet-bachelorproef}

%% TODO: Het is gebruikelijk aan het einde van de inleiding een overzicht te
%% geven van de opbouw van de rest van de tekst. Deze sectie bevat al een aanzet
%% die je kan aanvullen/aanpassen in functie van je eigen tekst.

De rest van deze bachelorproef is als volgt opgebouwd:

In Hoofdstuk~\ref{ch:methodologie} wordt de methodologie toegelicht en worden de gebruikte onderzoekstechnieken besproken om een antwoord te kunnen formuleren op de onderzoeksvragen.

%% TODO: Vul hier aan voor je eigen hoofstukken, één of twee zinnen per hoofdstuk

In Hoofdstuk~\ref{ch:conclusie}, tenslotte, wordt de conclusie gegeven en een antwoord geformuleerd op de onderzoeksvragen. Daarbij wordt ook een aanzet gegeven voor toekomstig onderzoek binnen dit domein.


%%=============================================================================
%% Methodologie
%%=============================================================================

\chapter{Methodologie}
\label{ch:methodologie}

%% TODO: Hoe ben je te werk gegaan? Verdeel je onderzoek in grote fasen, en
%% licht in elke fase toe welke stappen je gevolgd hebt. Verantwoord waarom je
%% op deze manier te werk gegaan bent. Je moet kunnen aantonen dat je de best
%% mogelijke manier toegepast hebt om een antwoord te vinden op de
%% onderzoeksvraag.

In dit hoofdstuk worden de methodes en denkpistes besproken die werden gehanteerd tijdens het opstellen van deze scriptie. Daarnaast wordt er reeds per hoofdstuk een inhoudelijk overzicht gegeven van wat de lezer kan verwachten bij het lezen van dit werk.

\section{Gehanteerde methodiek}

De bachelorproef is opgedeeld in twee grote onderdelen: een theoretisch deel en een meer praktisch gericht deel. 

In het theoretische gedeelte wordt er vooral gefocust op de interne werking van ZFS. Vooraleer echter de werking van ZFS uit te spitten, wordt er eerst een overzicht gegeven van RAID-systemen en RAID-niveaus. Nadien worden het ontwerp van ZFS en de beslissingen van de ontwikkelaars toegelicht; waar mogelijk wordt er telkens een vergelijking gemaakt met de manier waarop meer 'traditionele' oplossingen een bepaald probleem zouden aanpakken. Niet alle aspecten van de interne werking van ZFS worden besproken; daarvoor is deze scriptie ook helemaal niet bedoeld. Echter is een globaal beeld van de werking van ZFS van belang aangezien er toch wel significante verschillen zijn tussen een opslagstack binnen ZFS en een traditionele opslagstack. 

Het theoretisch deel biedt m.a.w. al grotendeels  een antwoord op de eerste twee onderzoeksvragen.

In het praktische gedeelte worden onder andere de performantie en betrouwbaarheid van ZFS geanalyseerd, om zo een antwoord te vinden op de laatste onderzoeksvraag. Bij dit onderdeel worden er twee testsystemen gebruikt, nl. een fysieke machine en een virtuele machine (via VirtualBox). Het eerstegenoemde systeem dient hoofdzakelijk om benchmarks uit te voeren; het tweede systeem dient uitsluitend om de betrouwbaarheid van een ZFS RAID-Z-opstelling na te gaan.

Voor het uitvoeren van de performantietesten wordt er gebruik gemaakt van \textbf{Phoronix Test Suite\footnote{\url{https://www.phoronix-test-suite.com}}}: deze suite is een wrapper rond veelgebruike benchmarktools en maakt het mogelijk om op een makkelijke manier relevante gegevens te verzamelen. Er is weinig tot geen voorafgaande kennis vereist voor het uitvoeren van de verschillende benchmarks, en dit was dan ook één van de hoofdredenen om voor dit programma te kiezen. 

Naast performantie en betrouwbaarheid, worden er ook nog andere aspecten van ZFS belicht, waaronder:

\begin{itemize}
  \item{Het voorbereiden en installeren van een computersysteem voor het gebruik van Linux en ZFS;}
  \item{Creatie en beheer van ZFS pools en VDEV's;}
  \item{De verschillende types van bestandssystemen (of datasets) die er binnen de ZFS stack bestaan;}
\end{itemize}

Hier en daar worden er nog enkele theoretische aspecten besproken, maar enkel al alleen als dit een toegevoegde waarde heeft. Bij bijvoorbeeld het hoofdstuk over VDEV's en storage pools is het noodzakelijk om te verduidelijken welke soorten VDEV's er bestaan; op deze manier wordt er context geschapen en is het voor de lezer ook duidelijker wat er in bepaalde gevallen bedoeld wordt.

\section{Opbouw van de bachelorproef}

Deze bachelorproef is verder globaal gezien als volgt opgebouwd:

%In Hoofdstuk \ref{ch:methodologie} wordt de methodologie toegelicht en worden de gebruikte onderzoekstech-
%nieken besproken om een antwoord te formuleren op de onderzoeksvragen.

In Hoofdstuk \ref{ch:h2} wordt er een korte inleiding gegeven op de geschiedenis en de algemene
werking van RAID-systemen. Ook ZFS en RAID-Z worden reeds kort toegelicht.

In Hoofdstuk \ref{ch:h3} wordt er een globaal overzicht gegeven van de architectuur en ontwerpprin-
cipes van ZFS. In de daaropvolgende hoofdstukken worden de belangrijkste onderdelen en
functionaliteiten wat meer uitgediept.

In Hoofdstuk \ref{ch:h4} wordt het opslagmodel van het ZFS-bestandssysteem besproken. Onder
andere de datastructuur en het transactiemodel van ZFS komen aan bod.

In Hoofdstuk \ref{ch:h5} worden de stappen die moeten worden ondernomen om een desktopcom-
puter om te zetten naar een Linux-server die kan worden gebruikt voor ZFS besproken.

In Hoofdstuk \ref{ch:h6} worden zpools en VDEV’s wat meer in detail belicht. Tevens wordt er
gedemonstreerd hoe men zpools en VDEV’s aanmaakt en wijzigt.

In Hoofdstuk \ref{ch:h7} worden traditionele bestandssystemen vergeleken met ZFS datasets. Onder andere de verschillende soorten datasets komen aan bod; tevens wordt er aan het eind van het hoofdstuk getoond hoe een ZFS dataset kan worden gebruikt om een NFS-share op te zetten.

In Hoofdstuk \ref{ch:h8} worden de prestaties van RAID-Z en Linux MD, een softwarematige RAID binnen Linux, met elkaar vergeleken. Hiervoor wordt er gebruik gemaakt van Phoronix Benchmark: dit is een wrapper rond verschillende onafhankelijke tools dat het verzamelen van relevante gegevens een stuk makkelijker maakt.

In Hoofdstuk \ref{ch:h9} wordt de betrouwbaarheid van ZFS nagegaan, met name: hoe regaeert een RAID-Z-opstelling op fouten onder verschillende omstandigheden?

In Hoofdstuk \ref{ch:conclusie}, tenslotte, wordt de conclusie gegeven en een antwoord geformuleerd op
de onderzoeksvragen. Daarbij wordt ook een aanzet gegeven voor toekomstig onderzoek
binnen dit domein.


%% Voeg hier je eigen hoofdstukken toe die de ``corpus'' van je bachelorproef
%% vormen. De structuur en titels hangen af van je eigen onderzoek. Je kan bv.
%% elke fase in je onderzoek in een apart hoofdstuk bespreken.

%%=============================================================================
%% H2 - Inleiding tot RAID \& ZFS
%%=============================================================================

\chapter{Inleiding tot RAID \& ZFS}
\label{ch:h2}

In dit hoofdstuk wordt er een korte inleiding gegeven over de basisprincipes van RAID, aangezien RAID-Z een softwarematige vorm van RAID is. Daarnaast wordt de geschiedenis en globale werking van ZFS reeds kort besproken. 

\section{Basisbeginselen van RAID}

\subsection{Geschiedenis}

Al van oudscher worden magnetische harde schijven gebruikt als opslagmedium voor data \autocite{Goda2012}. Maar reeds in de jaren 80 zagen onderzoekers in dat I/O-performance een \gls{bottleneck} zou vormen voor computersystemen in de toekomst: terwijl geheugenchips en processoren steeds sneller werden, bevonden opslagmedia zich in een impasse \autocite{DavidA.Paterson1987}.

In de paper \textit{"A Case for Redundant Arrays of Inexpensive Disks"}  formuleerden \textcite{DavidA.Paterson1987} en zijn collega's voor het eerst de term 'RAID', wat een acroniem is voor 'Redundant Arrays of Inexpensive Disks'. De oorspronkelijke idee achter RAID was dat een verzameling van goedkopere schijven performanter zou zijn dan grotere en duurdere mainframeschijven van die tijd. 

Naast \gls{performantie} en kost was \gls{betrouwbaarheid} (reliability) ook een belangrijke factor. Als bv. één of meerdere schijven van de array falen, dan mag dit geen invloed hebben op de werking van de rest van de verzameling schijven. Daarom introduceerden de onderzoekers de zgn. "RAID levels"    \autocite{DavidA.Paterson1987}, die vandaag de dag nog steeds in gebruik zijn. Er bestaan een aantal RAID-niveaus, waarvan de voornaamste zullen besproken worden.

\begin{figure}
	\centering
	\includegraphics[width=0.6\textwidth]{inleiding-raid-illustratie}
	\caption{Illustratie van verschillende RAID-niveaus \autocite{Chen1994}}
	\label{fig:chen_array_illustratie}
\end{figure}

\subsection{Eigenschappen van RAID-systemen}

Bij het bouwen van RAID-systemen worden er gebruikelijk drie aspecten in beschouwing genomen: \textbf{\gls{performantie}} (performance), \textbf{\gls{betrouwbaarheid}} (reliability) en \textbf{\gls{capaciteit}} (capacity). Een RAID-niveau is in principe niets anders dan een balans tussen deze verschillende eigenschappen; meestal zullen er dus één of meerdere trade-offs moeten gemaakt worden \autocite{Chen1994}.

Begrippen die centraal staan bij RAID zijn \textit{\textbf{\gls{striping}}} en \textit{\textbf{\gls{parity}}}. Striping heeft betrekking op de manier waarop een RAID-controller (hetzij hardwarematig, hetzij softwarematig) de blokken data verdeelt over de array van schijven. Bij RAID 0 bijvoorbeeld wordt de data gelijkmatig gedistribueerd volgens het \textit{round-robin}-algoritme \autocite{OSThreePiecesRemzi2015}. Datablokken die verdeeld zijn over meerdere schijven en samen één geheel vormen worden een \textit{stripe} genoemd. \\ 

\subsection{RAID-niveaus: een overzicht}

\subsubsection{RAID 0}

Een voordeel bij RAID 0 is dat de gehele \gls{capaciteit} van de schijven kan gebruikt worden; er gaat geen ruimte verloren, aangezien de data gelijkmatig verdeeld wordt over de array. Een bijkomend voordeel van striping is dat \gls{performantie} in het algemeen goed is \autocite{OSThreePiecesRemzi2015} : de meeste en reads en writes kunnen parallel worden afgehandeld. Een voorwaarde voor goede \gls{performantie} is echter wel dat de \textit{chunk size} (de grootte van de blokken data die worden weggeschreven en/of uitgelezen) ook optimaal wordt gekozen, i.e. afhankelijk van de workload op het systeem \autocite{OSThreePiecesRemzi2015}. Toch kent RAID 0 een groot nadeel: \gls{betrouwbaarheid} is nagenoeg onbestaande. Aangezien data nergens wordt gedupliceerd, betekent dat het falen van eender welke disk leidt tot verlies van data \autocite{OSThreePiecesRemzi2015}.

\subsubsection{RAID 4}

Parity is een mechanisme dat werd geïntroduceerd bij RAID-niveau 4 om \gls{betrouwbaarheid} af te dwingen. Bij RAID 4 wordt er metadata over de opgeslagen data bijgehouden in \gls{parity} blocks op een aparte schijf. Deze metadata wordt verkregen d.m.v. het uitvoeren van een mathematische functie op de opgeslagen data. Meestal is dit een XOR-functie (exclusieve OF) \autocite{Chen1994}. Aan de hand van deze \gls{parity} kan bij het verlies van één of meerdere schijven de originele data worden gereconstrueerd door XOR'ing toe te passen op de \gls{parity} bits en de data bits. Bij een XOR-operatie geven een even aantal enen (1) steeds als resultaat nul (0); omgekeerd geldt ook dat een oneven aantal enen (1) steeds een één (1) als resultaat zullen opleveren. Stel dat één schijf van een array van vier schijven faalt, dan kan nog steeds de originele data worden verkregen. Echter, als er meer dan één schijf verloren gaat, dan is het bij RAID 4 onmogelijk om de originele data te herstellen. Het grote voordeel van RAID 4 is dan weer echter dat er minder wordt ingeboet op \gls{capaciteit}  dan bij bv. RAID 1 en RAID 5 \autocite{OSThreePiecesRemzi2015}. 

\subsubsection{RAID 1}

Naast RAID 0 en RAID 4 zijn er nog andere RAID-levels, zoals RAID 1 en RAID 5. RAID 1 staat ook bekend als \textit{mirroring}, omdat het kopieën maakt van de datablokken naar één of meerdere disks afhankelijk van het aantal schijven. Op gebied van \gls{capaciteit} is RAID 1 niet echt gunstig, aangezien maar de helft van de totale schijfruimte bruikbaar is. Stel dat er vier schijven in een array aanwezig zijn, dan is slechts de opslagcapaciteit van twee schijven bruikbaar. Daarentegen is de \gls{betrouwbaarheid} van RAID 1 wel vele malen beter dan die van RAID 0: in theorie mogen er bij een reeks van \textit{n} schijven $\frac{n}{2}$ schijven falen. Maar dan mogen de schijven die elkaars mirror zijn niet falen, want dan is de data op deze disks verloren.  Daarom houdt men in de praktijk meestal de maatstaf van één schijf aan \autocite{OSThreePiecesRemzi2015}.

\subsubsection{RAID 5}

Als laatste wordt RAID 5 besproken. RAID 5 is in principe niets anders dan RAID-niveau 4, maar dan uitgebreid met functionaliteit dat de \gls{parity} blocks roteert over de verschillende schijven. Dit is een groot verschil t.o.v. RAID 4, waarbij de \gls{parity} blocks zich op één disk bevinden. Read-performantie is nagenoeg gelijk aan RAID 4, maar write-performantie is stukken beter. Dit komt omdat bij RAID 5 de schrijfoperaties parallel kunnen worden afgehandeld; bij RAID 4 vormt de parity-schijf een \gls{bottleneck} bij het wegschrijven van data \autocite{Chen1994}. De reden hiervoor is dat bij het updaten van data ook de \gls{parity} blocks moeten worden geüpdatet; alle operaties worden dus m.a.w. serieel uitgevoerd \autocite{OSThreePiecesRemzi2015}.

Er bestaan nog andere, niet-standaard RAID levels, zoals RAID 6 en RAID 10, maar deze worden hier niet besproken.

\section{Inleiding tot ZFS \& RAID-Z}

\subsection{Geschiedenis}

ZFS is een bestandssysteem dat ontwikkeld is door het toenmalige Sun Microsystems, nu onderdeel van Oracle Corporation. Voormalig Sun-werknemer Jeff Bonwick was de oorspronkelijke hoofdontwikkelaar van het bestandssysteem. De ontwikkeling van ZFS startte aan het begin van de jaren 2000; Sun had reeds met de ontwikkeling van bestandssystemen geëxperimenteerd, maar deze pogingen mislukten telkens \autocite{Bonwick2015}. ZFS is een \textit{copy-on-write} (COW) bestandssysteem \autocite{BrianHickmann2007}: bij elke aanpassing van een datablok, wordt het datablok in kwestie niet aangepast, maar wordt het gekopieerd naar een nieuwe locatie en aangepast \autocite{Lucas2015}.

In die tijd gebruikte het bedrijf haar UNIX-besturingssysteem Solaris intern voor verschillende soorten toepassingen, waaronder file servers \autocite{Bonwick2015}. De schijven van deze servers waren opgedeeld in volumes, beheerd door de Solaris Volume Manager (SVM) en geformatteerd in UFS (UNIX Filesystem) \autocite{Bonwick2015}. Toen een verkeerd ingevoerd SVM-commando erin slaagde om het systeem te doen crashen en voor een enorme \textit{downtime} zorgde bij  Sun, was dit voor Bonwick een aanleiding om volledig \textit{from scratch} een bestandssysteem op te bouwen dat makkelijk in gebruik én beheer was.

De hoofdreden om een volledig nieuw bestandssysteem te ontwikkelen was volgens \textcite{JeffBonwick_lastZFS} dat de toenmalige oplossingen voor bestandssystemen totaal achterhaald waren. Bestandssystemen waren nog ontwikkeld voor opslagnoden uit de jaren '80 en '90, welke niet te vergelijken zijn met de huidige noden van zowel bedrijven als particulieren. Naarmate de nood aan meer opslagruimte steeg, moesten er oplossingen worden bedacht, en dit m.b.v. bestandssystemen die hier helemaal niet op voorzien waren. Een 'tussenoplossing', aldus volgens \textcite{JeffBonwick_lastZFS}, die ook vandaag de dag nog gebruikt wordt, is de combinatie van volumes en bestandssystemen. Volumes zijn abstracties voor (delen van) fysieke schijven.

%Een andere eigenaardigheid van bestandssystemen is dat deze nog steeds onlosmakelijk zijn verbonden aan een (deel van) een schijf of opslagapparaat. Volume managers zorgen wel voor een zeker niveau van abstractie, maar bestandssystemen blijven nog steeds gekoppeld aan een bepaalde reeks blokken \autocite{ZFSBonwick}. Dit vonden de ontwikkelaars van ZFS nogal omslachtig: zij zijn van mening dat een bestandssysteem een virtuele abstractie van opslagruimte vormt en dus los moet staan van de fysieke blokken op de schijf. \textcite{ZFSBonwick} maakt de vergelijking met het adresseren van RAM geheugen: RAM-geheugen moet ook eerst niet worden geformatteerd en dan apart worden toegewezen aan applicaties. Bij het alloceren en aanspreken van RAM-geheugen is het immers niet nodig voor de applicatie of programmeur om het exacte geheugenadres te kennen; deze taken worden al afgehandeld door memory allocators m.b.v. virtueel geheugen \autocite{ZFSBonwick}.

%Daarom stelden de ZFS-ontwikkelaars \textit{pooled storage} voor, samen met een geïntegreerde volume manager en storage allocator. Een bijkomend voordeel van \textit{pooled storage} is dat bestandssystemen dynamisch kunnen groeien en verkleinen, zonder tussenkomst van de gebruiker, aangezien de opslagruimte van alle disks als één geheel wordt gezien i.p.v. afzonderlijke eenheden. ZFS maakt het wel mogelijk om quota's in te stellen op bestandssystemen, opdat één bestandssystemen niet alle beschikbare ruimte zou innemen \autocite{ZFSBonwick}. 

\subsection{RAID-Z}

Het ZFS-bestandssysteem omvat ook een softwarematige RAID, RAID-Z genaamd. Volgens \textcite{Bonwick2005} lost RAID-Z in combinatie met ZFS een aantal problemen op die inherent aanwezig zijn bij andere RAID-implementaties. Zo claimen de ontwikkelaars dat RAID-Z het zogenaamde RAID5 \textit{write hole} probleem volledig oplost. Bij klassieke RAID-oplossingen worden stripes weggeschreven op een niet-atomaire manier, i.e. de bewerkingen worden niet als één geheel uitgevoerd. Een bijkomend probleem hierbij is dat bij RAID-levels met \gls{parity} (zoals RAID5) ook nog eens de parity-blocks moeten herberekend worden. Indien het systeem tussen deze bewerkingen crasht of uitvalt door bv. elektriciteitsproblemen, dan bevindt de data op de array van schijven zich in een inconsistente toestand: stripes en \gls{parity} blocks kunnen corrupt raken \autocite{Bonwick2005}. 

%Een oplossing voor dit probleem is het gebruik van NVRAM waarin stripes tijdelijk kunnen worden bijgehouden. Indien er zich dan een systeemcrash voordoet, dan kan de RAID-controller m.b.v. de gegevens in het NVRAM de originele data herstellen. Maar waar RAID geen oplossing voor heeft, aldus volgens \textcite{Bonwick2005}, is het optreden van \textit{silent data corruption}; de RAID-controller weet immers niet of het al dan niet om corrupte data gaat. Het enige waar een RAID-controller zicht op heeft, zijn blokken. Hiervoor biedt ZFS samen met RAID-Z een oplossing: aangezien het bestandssysteem en RAID-Z samen één geheel vormen, is het mogelijk om van datacorruptie te herstellen of om dit zelfs tegen te gaan. Bij het herstellen van data op de array, overloopt RAID-Z de metadata van het bestandssysteem om zo de stripe-grootte te achterhalen. Ook vergelijkt ZFS bij elk datablok de berekende checksum: indien deze checksums niet overeenkomen, dan tracht ZFS dit blok te repareren m.b.v. de parity-informatie van RAID-Z. Op deze manier kan datacorruptie vroegtijdig worden opgemerkt en kan defecte hardware eventueel vervangen worden \autocite{Bonwick2005}.

\subsection{Toekomst van ZFS}

Ondertussen werd Sun Microsystems overgenomen door Oracle in 2010, na het faillissement van deze eerstegenoemde \autocite{OracleOnbekend}. In 2010 richtten enkele ex-ontwikkelaars van Solaris het illumos-project op met de bedoeling om een volledig open source variant van OpenSolaris te ontwikkelen; ook ZFS werd verder ontwikkeld als onderdeel van illumos \autocite{illumos2012}. De reden hiervoor was dat Oracle geen nieuwe uitgaven meer uitbracht voor OpenSolaris. In 2013 werd het OpenZFS-project opgericht, dat zichzelf als de \textit{"ware en open opvolger van het oorspronkelijke ZFS-project"} ziet \autocite{OpenZFSHistory2014}.

Mede dankzij dit project is ZFS ondertussen beschikbaar op verschillende platformen, waaronder FreeBSD, Linux en macOS.



%%=============================================================================
%% H3 - Ontwerpprincipes van ZFS
%%=============================================================================

\chapter{Ontwerpprincipes \& architectuur van ZFS}
\label{ch:h3}

In dit hoofdstuk worden enkele principes besproken waarop de ontwikkelaars zich hebben gebaseerd bij het ontwerp en de onwtikkeling van ZFS. Tevens wordt de architectuur van ZFS globaal beschreven, om zo de ontwerpbeslissingen van de ontwikkelaars wat meer toe te lichten.

\section{Ontwerpprincipes}

De principes die aan de basis van ZFS liggen vloeiden meestal voort uit de problemen die de ontwikkelaars zelf ervaarden bij het gebruik van andere bestandssystemen.

\subsection{Eenvoud van beheer \& Storage Pools}

Volgens \textcite{ZFSBonwick} kan en moet het aanmaken en beheren van bestandssystemen een stuk makkelijker gemaakt worden. Hierbij speelt automatisatie van verschillende taken een belangrijke rol. Ook moet het mogelijk zijn om beheerderstaken (zoals bestandssystemen aanmaken en verwijderen) uit te voeren zonder de werking van het gehele systeem te ondermijnen \autocite{ZFSBonwick}. 

Een niet onbelangrijke feature hierbij zijn storage pools. Storage pools hebben als doel om opslagruimte zoveel mogelijk los te koppelen van de fysieke schijven: alle schijven bevinden zich in een pool van disks. Uit deze pool kunnen bestandssystemen worden aangemaakt, zonder rekening te moeten houden met de limitaties van bv. partities. Tevens kunnen bestandssystemen op een flexibele manier gebruik maken van deze pooled storage door automatisch in te krimpen en uit te breiden wanneer nodig \autocite{ZFSBonwick}. 

\begin{figure}
        \centering
        \includegraphics[width=0.8\textwidth]{h3-pools-vs-vols}
        \caption{Illustratie van ZFS pooled storage (rechts) t.o.v.volume-based storage (links) \autocite{ZFSBonwick}}
        \label{fig:bonwick_pools_illustratie}
\end{figure}
  
\subsection{Consistentie \& Integriteit}

Eén van de taken van een bestandssysteem is om ervoor te zorgen dat het in een consistente toestand blijft, i.e. het moet mogelijk zijn om van inconsistente toestanden te herstellen d.m.v. bijvoorbeeld een journal of door een filesystem check te draaien \autocite{OSThreePiecesRemzi2015}. Een nadeel aan deze technieken volgens \textcite{ZFSBonwick} is dat deze niet makkelijk zijn om te implementeren, omdat bijvoorbeeld in het geval van journaling filesystems een roll back of roll forward moet worden uitgevoerd. Hierbij moet ook nog de volgorde van de bewerkingen in acht worden genomen \autocite{OSThreePiecesRemzi2015}. 

De oplossing volgens \textcite{ZFSBonwick} is om het bestandssysteem te allen tijde consistent te houden; er mag m.a.w. geen enkel moment zijn waarop het systeem in een inconsistente toestand kan terecht komen. Dit wordt verwezenlijkt door de Copy-On-Write eigenschappen van ZFS, waardoor bewerkingen atomair gebeuren \autocite{Li2009}. 

Een ander aspect waar ZFS een antwoord op tracht te vinden, is het voorkomen en minimaliseren van datacorruptie. Volgens \textcite{ZFSBonwick} is het niet verstandig om volledig te vertrouwen op hardware, omdat er steeds een kans is dat deze bugs bevat. Dit kan leiden tot \textit{silent data corruption}: dit is een fenomeen waarbij de schijf corrupte datablokken niet kan detecteren \autocite{OSThreePiecesRemzi2015}. Om deze reden bevat ZFS een uitgebreid checksumming-mechanisme dat werkt op blokniveau. Indien men ZFS zou voorstellen als een boomstructuur, dan bevatten de \textit{parent blocks} steeds de checksums van hun kinderen \autocite{ZFSBonwick}. Bij het uitlezen en wegschrijven van een blok data wordt de checksum berekend en gecontroleerd met de (eventueel) reeds bestaande checksum. Indien deze overeenkomen, dan wordt het datablok doorgegeven. Indien deze niet overeenkomen, dan tracht ZFS het corrupte blok te repareren. Echter moet er wel opgemerkt worden dat er voor automatische foutcorrectie (goede) kopieën van de corrupte blokken moeten beschikbaar zijn \autocite{Li2009}.

\subsection{Ingebouwde volumebeheerder}

Indien bijvoorbeeld LVM (Logical Volume Manager) wordt gebruikt op een Linux-systeem voor het aanmaken van volumes, dan staan de volumes relatief "los"  van de bestandssystemen die worden aangemaakt op deze volumes: eerst maakt men volumes aan, nadien pas bestandssystemen \autocite{Lewis2006}. 

\textcite{ZFSBonwick} is van mening dat een volledige integratie van de volumebeheerder in het bestandssysteem een aantal interessante voordelen met zich meebrengt, zoals betere optimalisaties en betere consistentie, omdat de volume manager nu "weet" hoe de bestandssystemen in de storage pool(s) zijn opgebouwd.

\section{De architectuur van ZFS: een overzicht}

De architectuur van ZFS is opgebouwd uit een zestal componenten \autocite{Li2009}: de Storage Pool Allocator (SPA), de Data Management Unit (DMU), een ZFS POSIX Layer (ZPL), de ZFS Attribute Processor (ZAP), de ZFS Intent Log (ZIL) en ZFS Volume (ZVOL). Elk onderdeel biedt een bepaalde functionaliteit aan en levert diensten aan boven- en onderliggende lagen.

\subsection{Storage Pool Allocator (SPA)}

De taak van de SPA is hoofdzakelijk om datablokken van verschillende apparaten in één pool te verzamelen; de functionaliteit van de SPA komt m.a.w. in grote mate overeen met die van een volume manager. Het grote verschil tussen een volume manager en de SPA is dat de SPA louter een \textit{interface} is voor virtuele datablokken; dit in tegenstelling tot een volume manager, welke wordt voorgesteld door een block device \autocite{ZFSBonwick}. 

De Storage Pool Allocator biedt een interface aan tot \textbf{data virtual addresses (DVA's)}: dit zijn de adressen van de virtuele datablocks die zich in een storage pool bevinden. De datablokken die zich fysiek op een schijf bevinden worden dus niet rechtstreeks aangesproken; alle communicatie van bovenliggende lagen gericht tot de fysieke disks moet eerst via de SPA. Hierdoor worden de ontwerpprincipes van \textit{'Eenvoud van beheer'} en \textit{'Storage Pools'} gerealiseerd: schijven kunnen dynamisch worden toegevoegd zonder de werking van de rest van het systeem te ondermijnen. Ook kan de systeembeheerder op een flexibele manier bestandssystemen aanmaken, zonder rekenschap te moeten geven aan de onderliggende fysieke structuur \autocite{ZFSBonwick}.

\subsection{Data Management Unit (DMU)}

De Data Management Unit is de laag boven de SPA en is a.h.w. de 'lijm' tussen de nogal \textit{low-level} (virtuele) datablokken en de bovenliggende lagen. Deze component vertaalt de blokken naar ZFS objecten; dit is nodig omdat ZFS alles voorstelt als objecten. Zo stelt een ZFS-object van het type \texttt{DMU\_OT\_ACL} een access control list voor \autocite{Microsystems2006}. Objecten behoren steeds tot een bepaalde naamruimte (\textit{namespace}) of dataset; dit verhoogt de flexibiliteit m.b.t. het beheer van bestandssystemen aanzienlijk. Zo wordt een bestandssysteem binnen ZFS voorgesteld door objecten uit een bepaalde dataset. Aangezien objecten van elkaar gescheiden zijn door private namespaces, leven bestandssystemen (die uit een reeks van ZFS-objecten bestaan) ook naast en onafhankelijk van elkaar. Dit vergemakkelijkt taken zoals aanmaken en verwijderen van filesystems en stuk makkelijker voor zowel de gebruiker als de DMU \autocite{ZFSBonwick}   

Daarnaast staat de DMU ook garant voor de consistentie van alle data. Om deze reden worden transacties binnen ZFS geïmplementeerd door de Data Management Unit als COW-transacties (Copy-On-Write). Hierbij wordt gebruik gemaakt van een 'alles of niets'-aanpak: ofwel is de transactie gelukt, ofwel is de transactie mislukt en moeten er back-upblokken worden aangewend \autocite{ZFSBonwick}.

\subsection{ZFS POSIX Layer (ZPL)}

Deze laag vertaalt objecten komende van de Data Management Unit naar met POSIX compatibele bestandssystemen \autocite{ZFSBonwick}. POSIX (acroniem voor Portable Operating System Interface) is een reeks van standaarden waaraan ontwikkelaars zich moeten houden indien ze willen dat hun programma's \textit{portable} (overdraagbaar) zijn. Dit betekent dat het weinig tot geen moeite zou moeten kosten voor een ontwikkelaar om een programma draaiende te krijgen op bv. verschillende UNIX-systemen \autocite{IEEE2016}. MacOS en Solaris zijn voorbeelden van 'POSIX-compliant' besturingssystemen \autocite{GroupOnbekend}.

Voor het uitvoeren van bewerkingen maakt de ZPL gebruik van de onderliggende Data Management Unit. Het aanmaken van een bestandssysteem gebeurt bijvoorbeeld door de ZFS POSIX Layer: de ZPL maakt enkele DMU-objecten aan metadata e.d. aan en maakt gebruik van het transactiemodel van de DMU om deze objecten weg te schrijven. Dankzij deze aanpak verlopen deze bewerkingen ook atomair \autocite{ZFSBonwick}. 

\subsection{ZFS Attribute Processor (ZAP)}

De ZFS Attribute Processor is een component die samenwerkt met de DMU-laag. De ZAP manipuleert ZAP-objecten: dit zijn speciale DMU-objecten die metadata over allerhande dingen bijhouden in de vorm van key-value pairs. Deze informatie handelt over verschillende andere objecten, zoals datasets en filesystem objects. Voor een grote hoeveelheid attributen (en dus informatie) wordt er gebruikgemaakt van zgn. fatzaps; indien de hoeveelheid informatie nogal gering is, dan kies ZFS voor microzaps \autocite{Microsystems2006}.

\subsection{ZFS Intent Log (ZIL)}

Deze component houdt logbestanden bij van alle transacties voor het geval dat het bestandssysteem toch in een inconsistente toestand zou terecht komen. (HIER ZAT IK) 



%%=============================================================================
%% H4 - Het opslagmodel van ZFS
%%=============================================================================

\chapter{Het opslagmodel van ZFS}
\label{ch:h4}

In dit hoofdstuk worden de interne bestandssysteemoperaties van ZFS wat meer toegelicht. Hierbij wordt vooral de werking van de Storage Pool Allocator en de Data Management Unit dieper uitgespit, aangezien deze componenten het beheer van data voor zich nemen.

\section{Structuur van het bestandssysteem}

Datablokken worden in ZFS voorgesteld als een boomstructuur. Vele andere bestandssystemen, zoals BTRFS op Linux, gebruiken ook boomstructuren voor het bijhouden van data \autocite{Project2017a}. De algemene werking van ZFS en andere, boomgebaseerde bestandssystemen verschillen niet zo heel veel. Echter zijn er ook eigenschappen die uniek zijn aan de manier waarop ZFS de data opslaat.

De belangrijkste elementen van een boom in de informatica zijn de volgende: de wortel (Eng.: \textit{root}), de knopen (Eng.: \textit{nodes}) en de bladeren (Eng.: \textit{leaves}) \autocite{Cohen}. Indien men bij de wortel van een ZFS-boom start, dan komt men als eerste de \textit{\"{u}berblock} tegen: dit is simpelweg een andere benaming voor de wortel van de boom. Dit datablok bevat een \gls{checksum} van zichzelf en verwijst naar de andere datablokken in de boom. Afhankelijk van de sectorgrootte van een schijf, bevat een ZFS pool een zeker aantal \"{u}berblocks (een schijf waarvan de sectoren 512 bytes groot zijn geeft 128 \"{u}berblocks als resultaat). Doorgaans hebben de meeste datablokken geen vaste blokgrootte; ZFS probeert de grootte van blokken zo goed mogelijk aan te passen aan de data die moet worden opgeslagen \autocite{Lucas2015}.

Zoals reeds gezegd in Hoofdstuk \ref{ch:h3}, worden alle blokken gechecksummed om corruptie van data tegen te gaan. Deze checksums worden bewaard in het datablok zelf en in de ouder van dit blok: op deze manier is de hele ketting van datablokken gechecksummed en kan er makkelijk schade worden vastgesteld en kan deze schade eventueel worden gerepareerd. De \textit{\"{u}berblock} is het enige datablok die geen ouder heeft, en dus bewaart deze zijn \gls{checksum} bij zichzelf \autocite{ZFSBonwick}.  

\section{Checksumming \& Redundantie op blokniveau}

Checksumming is een goede oplossing om corruptie te detecteren. Echter moeten er nog steeds back-upblokken aanwezig zijn om van datacorruptie te herstellen. Indien men bijvoorbeeld mirroring toepast binnen een pool, dan kan ZFS de corrupte datablokken zelf herstellen door een goede kopie van de andere schijf van de mirror te halen. Als een applicatie een aanvraag doet voor een reeks datablokken waarvan er één of meerdere beschadigd zijn, dan repareert ZFS deze datablokken automatisch; nadien worden de herstelde blokken doorgegeven aan de applicatie, zonder dat deze laatstgenoemde hier iets van merkt \autocite{ZFSBonwick}.

Maar zelfs zonder gebruik te maken van mirror VDEV's kan ZFS in sommige gevallen van corruptie herstellen. Hiervoor maakt ZFS gebruik van zogenaamde \textit{ditto blocks}: dit zijn simpelweg kopieeën van belangrijke datablokken, zoals van metadata of pool data. Deze blokken worden zo ver mogelijk uit elkaar op de schijf geplaatst, om zo de impact van datacorruptie te minimaliseren \autocite{Lucas2015}. 

\section{Transacties binnen ZFS}

ZFS is een transactioneel bestandssysteem \autocite{ZFSBonwick}. Transacties worden binnen ZFS afgehandeld door de DMU\footnote{De DMU behandelt eigenlijk enkel objecten, maar voor de eenvoud wordt er in dit hoofdstuk meestal gesproken over blokken.} en verlopen volgens een Copy-On-Write (COW) manier. Bij Copy-On-Write worden blokken nooit direct aangepast of overschreven; in plaats daarvan kopieert ZFS deze blokken eerst naar een andere locatie, om deze dan vervolgens aan te passen. Deze manier van van werken garandeert dat data op de schijf of schijven steeds consistent blijft: als er zich bijvoorbeeld een stroompanne voordoet midden in een transactie, dan kan ZFS makkelijk terugvallen op de oude boomstructuur, aangezien deze niet wordt overschreven \autocite{Lucas2015}.

\begin{figure}
        \centering
        \includegraphics[width=1.0\textwidth]{h4_cow}
        \caption{Boomstructuur bij aanpassing van één datablok binnen een ZFS-transactie. Van links naar rechts: (1) Aanpassen van het datablok; (2) Aanpassen van de indirecte blokken; (3) Overschrijven van de \"{u}berblock. Alle bewerkingen gebeuren op een COW-manier. \autocite{ZFSBonwick}}
        \label{fig:bonwick_cow_illustratie}
\end{figure}

Uit de paper van \textcite{ZFSBonwick} kan men de volgende stappen opmaken die worden ondernomen bij het uitvoeren van een transactie:

\begin{enumerate}
  \item{De ZFS POSIX Layer (ZPL) krijgt de opdracht om een reeks van blokken aan te passen, te creeëren of te verwijderen. Stel dat er in dit geval één enkel datablok moet worden aangepast en dat dit blok zich in een blad van de boom bevindt. De ZPL groepeert alle wijzigingen in een transaction group (txg); hierbij moet de ZPL de wijzigingen goed groeperen opdat het bestandssysteem aan het einde van de transactie in een consistente toestand wordt achtergelaten. Voor het uitvoeren van de transactie doet de ZPL beroep op de DMU.}
  \item{De DMU start de transactie en doorzoekt de boom. De DMU heeft het blok in kwestie gevonden, kopieert het naar een nieuwe locatie en past het vervolgens aan (Copy-On-Write). Na deze aanpassing moet ook de \gls{checksum} van het datablok worden herberekend.}
  \item{Aangezien andere blokken in de boom direct of indirect gekoppeld zijn aan het aangepaste datablok, moeten deze ook worden aangepast, en dit op dezelfde manier als het reeds aangepaste datablok. Ook moeten de checksums van deze blokken herberekend worden. Dit is een recursief gebeuren, en stopt bij de wortel van de boom, nl. de \"{u}berblock. }
  \item{Om de COW-eigenschappen van ZFS te garanderen, wordt het \"{u}berblock niet direct aangepast, maar wordt er geroteerd naar de volgende beschikbare \"{u}berblock. De oude en nieuwe boomstructuren worden dus a.h.w. met elkaar omgewisseld, en dit op een atomaire, transactionele manier.}
\end{enumerate}

Indien het systeem zou crashen tijdens deze transactie, dan kan de \"{u}berblock corrupt raken. Aangezien corruptie kan worden gedetecteerd met behulp van checksums, kan er eenvoudig worden teruggevallen op een oudere versie van een \"{u}berblock \autocite{ZFSBonwick}.





%%=============================================================================
%% Conclusie
%%=============================================================================

\chapter{Conclusie}
\label{ch:conclusie}

%% TODO: Trek een duidelijke conclusie, in de vorm van een antwoord op de
%% onderzoeksvra(a)g(en). Wat was jouw bijdrage aan het onderzoeksdomein en
%% hoe biedt dit meerwaarde aan het vakgebied/doelgroep? Reflecteer kritisch
%% over het resultaat. Had je deze uitkomst verwacht? Zijn er zaken die nog
%% niet duidelijk zijn? Heeft het ondezoek geleid tot nieuwe vragen die
%% uitnodigen tot verder onderzoek?

%\lipsum[76-80]

Doorheen deze bachelorproef is het (hopelijk) duidelijk geworden dat het gebruik van de term 'bestandssysteem' ZFS eigenlijk een beetje onrecht aandoet. De afkorting 'ZFS' mag dan wel Zettabyte Filesystem betekenen, toch is ZFS veel meer dan een bestandssysteem alleen: het is een volledig nieuwe implementatie van de traditionele opslagstack zoals we die al jaren gewend zijn. Daar waar meer traditionele oplossingen vaak bestaan uit verschillende, losse onderdelen die met elkaar samenwerken, bestaat de ZFS storage stack uit een aantal nauw samenwerkende lagen. Toch voelt dit niet aan als een groot, log geheel; functionaliteiten zijn duidelijk afgebakend en dit weerspiegelt zich ook in het gebruik en beheer.   

Toen ZFS voor het eerst op het toneel verscheen, waren vele techneuten laaiend enthousiast: problemen zoals datacorruptie en inflexibele opslag zouden eindelijk van de baan zijn.  


%%---------- Back matter ------------------------------------------------------

\printbibliography
\addcontentsline{toc}{chapter}{\textcolor{maincolor}{\IfLanguageName{dutch}{Bibliografie}{Bibliography}}}

% Woordenlijst
\printnoidxglossaries

\listoffigures
\listoftables

\end{document}
