%%=============================================================================
%% LaTeX sjabloon voor bachelorproef, HoGent Bedrijf en Organisatie
%% Opleiding Toegepaste Informatica
%%=============================================================================

\documentclass[fleqn,a4paper,12pt]{book}

%%=============================================================================
%% LaTeX sjabloon voor de bachelorproef, HoGent Bedrijf en Organisatie
%% Opleiding toegepaste informatica
%%
%% Structuur en algemene vormgeving. Meestal hoef je hier niets te wijzigen.
%%
%% Vormgeving gebaseerd op "The Legrand Orange Book", version 2.0 (9/2/15)
%% door Mathias Legrand (legrand.mathias@gmail.com) met aanpassingen door
%% Vel (vel@latextemplates.com). Het oorspronkelijke template is te vinden op
%% http://www.LaTeXTemplates.com
%%
%% Aanpassingen voor HoGent toegepaste informatica: 
%%   Bert Van Vreckem <bert.vanvreckem@hogent.be>
%% Licentie: 
%%   CC BY-NC-SA 3.0 (http://creativecommons.org/licenses/by-nc-sa/3.0/)
%%=============================================================================

%%-----------------------------------------------------------------------------
%% Packages
%%-----------------------------------------------------------------------------

\usepackage[top=3cm,bottom=3cm,left=3cm,right=3cm,headsep=10pt,a4paper]{geometry} % Page margins
\usepackage[utf8]{inputenc}  % Accenten gebruiken in tekst (vb. é ipv \'e)
\usepackage{amsfonts}        % AMS math packages: extra wiskundige
\usepackage{amsmath}         %   symbolen (o.a. getallen-
\usepackage{amssymb}         %   verzamelingen N, R, Z, Q, etc.)
\usepackage[english,dutch]{babel}    % Taalinstellingen: woordsplitsingen,
                             %  commando's voor speciale karakters
                             %  ("dutch" voor NL)
\usepackage{iflang}
\usepackage{eurosym}         % Euro-symbool €
\usepackage{geometry}
\usepackage{graphicx}        % Invoegen van tekeningen
\graphicspath{{img/}}       % Specifies the directory where pictures are stored
\usepackage{tikz}            % Required for drawing custom shapes
\usepackage[pdftex,bookmarks=true]{hyperref}
                             % PDF krijgt klikbare links & verwijzingen,
                             %  inhoudstafel
\usepackage{enumitem}        % Customize lists
\setlist{nolistsep}         % Reduce spacing between list items
\usepackage{listings}        % Broncode mooi opmaken
\usepackage{multirow}        % Tekst over verschillende cellen in tabellen
\usepackage{rotating}        % Tabellen en figuren roteren

\usepackage{booktabs}        % Required for nicer horizontal rules in tables

\usepackage{xcolor}          % Required for specifying colors by name
\definecolor{maincolor}{RGB}{0,147,208} % Define the main color used for 
                             % highlighting throughout the book
                             % 0, 147, 208 = officiële kleur HoGent FBO

% Paragraph style: no indent, add space between paragraphs
\setlength{\parindent}{0em}
\setlength{\parskip}{1em}

\usepackage{etoolbox}
\usepackage{titling} % Macros for title, author, etc
\usepackage{lipsum}          % Voor vultekst (lorem ipsum)

% Woordenlijst
\usepackage[toc]{glossaries}

% Tree
\usepackage{qtree}

\lstdefinestyle{command_style}{
  basicstyle=\ttfamily\footnotesize,
  breaklines=true,
  showstringspaces=false
}

% SVG
\usepackage{svg}

%----------------------------------------------------------------------------------------
%	FONTS
%----------------------------------------------------------------------------------------

\usepackage{avant} % Use the Avantgarde font for headings
%\usepackage{times} % Use the Times font for headings
\usepackage{mathptmx} % Use the Adobe Times Roman as the default text font together with math symbols from the Sym­bol, Chancery and Com­puter Modern fonts

\usepackage{microtype} % Slightly tweak font spacing for aesthetics
\usepackage[utf8]{inputenc} % Required for including letters with accents
\usepackage[T1]{fontenc} % Use 8-bit encoding that has 256 glyphs

%------------------------------------------------------------------------------
%	TITLE PAGE
%------------------------------------------------------------------------------

\newcommand{\inserttitlepage}{%
\begin{titlepage}
  \newgeometry{top=2cm,bottom=1.5cm,left=1.5cm,right=1.5cm}
  \begin{center}

    \begingroup
    \rmfamily
    \includegraphics[width=2.5cm]{img/HG-beeldmerk-woordmerk}\\[.5cm]
    Faculteit Bedrijf en Organisatie\\[3cm]
    \titel
    \vfill
    \student\\[3.5cm]
    Scriptie voorgedragen tot het bekomen van de graad van\\professionele bachelor in de toegepaste informatica\\[2cm]
    Promotor:\\
    \promotor\\
    \ifdefempty{\copromotor}{\vspace{2.5cm}}{Co-promotor:\\\copromotor\\[2.5cm]}
    Instelling: \instelling\\[.5cm]
    Academiejaar: \academiejaar\\[.5cm]
    \ifcase \examenperiode \or Eerste \or Tweede \else Derde \fi examenperiode
    \endgroup

  \end{center}
  \restoregeometry
\end{titlepage}
  \emptypage
\begin{titlepage}
  \newgeometry{top=5.35cm,bottom=1.5cm,left=1.5cm,right=1.5cm}
  \begin{center}

    \begingroup
    \rmfamily
    \IfLanguageName{dutch}{Faculteit Bedrijf en Organisatie}{Faculty of Business and Information Management}\\[3cm]
    \titel
    \vfill
    \student\\[3.5cm]
    \IfLanguageName{dutch}{Scriptie voorgedragen tot het bekomen van de graad van\\professionele bachelor in de toegepaste informatica}{Thesis submitted in partial fulfillment of the requirements for the degree of\\professional bachelor of applied computer science}\\[2cm]
    Promotor:\\
    \promotor\\
    \ifdefempty{\copromotor}{\vspace{2.5cm}}{Co-promotor:\\\copromotor\\[2.5cm]}
    \IfLanguageName{dutch}{Instelling}{Institution}: \instelling\\[.5cm]
    \IfLanguageName{dutch}{Academiejaar}{Academic year}: \academiejaar\\[.5cm]
    \IfLanguageName{dutch}{%
    \ifcase \examenperiode \or Eerste \or Tweede \else Derde \fi examenperiode}{%
    \ifcase \examenperiode \or First \or Second \else Third \fi examination period}
    \endgroup

  \end{center}
  \restoregeometry
\end{titlepage}
}

%----------------------------------------------------------------------------------------
%	BIBLIOGRAPHY AND INDEX
%----------------------------------------------------------------------------------------

\usepackage[style=apa,backend=biber]{biblatex}
\usepackage{csquotes}
\DeclareLanguageMapping{dutch}{dutch-apa}
\addbibresource{bachproef-tin.bib} % BibTeX bibliography file
\defbibheading{bibempty}{}

\usepackage{calc} % For simpler calculation - used for spacing the index letter headings correctly
\usepackage{makeidx} % Required to make an index
\makeindex % Tells LaTeX to create the files required for indexing

%----------------------------------------------------------------------------------------
%	MAIN TABLE OF CONTENTS
%----------------------------------------------------------------------------------------

\usepackage{titletoc} % Required for manipulating the table of contents

\contentsmargin{0cm} % Removes the default margin

% Part text styling
\titlecontents{part}[0cm]
{\addvspace{20pt}\centering\large\bfseries}
{}
{}
{}

% Chapter text styling
\titlecontents{chapter}[1.25cm] % Indentation
{\addvspace{12pt}\large\sffamily\bfseries} % Spacing and font options for chapters
{\color{maincolor!60}\contentslabel[\Large\thecontentslabel]{1.25cm}\color{maincolor}} % Chapter number
{\color{maincolor}}
{\color{maincolor!60}\normalsize\;\titlerule*[.5pc]{.}\;\thecontentspage} % Page number

% Section text styling
\titlecontents{section}[1.25cm] % Indentation
{\addvspace{3pt}\sffamily\bfseries} % Spacing and font options for sections
{\contentslabel[\thecontentslabel]{1.25cm}} % Section number
{}
{\hfill\color{black}\thecontentspage} % Page number
[]

% Subsection text styling
\titlecontents{subsection}[1.25cm] % Indentation
{\addvspace{1pt}\sffamily\small} % Spacing and font options for subsections
{\contentslabel[\thecontentslabel]{1.25cm}} % Subsection number
{}
{\ \titlerule*[.5pc]{.}\;\thecontentspage} % Page number
[]

% List of figures
\titlecontents{figure}[0em]
{\addvspace{-5pt}\sffamily}
{\thecontentslabel\hspace*{1em}}
{}
{\ \titlerule*[.5pc]{.}\;\thecontentspage}
[]

% List of tables
\titlecontents{table}[0em]
{\addvspace{-5pt}\sffamily}
{\thecontentslabel\hspace*{1em}}
{}
{\ \titlerule*[.5pc]{.}\;\thecontentspage}
[]

%----------------------------------------------------------------------------------------
%	MINI TABLE OF CONTENTS IN PART HEADS
%----------------------------------------------------------------------------------------

% Chapter text styling
\titlecontents{lchapter}[0em] % Indenting
{\addvspace{15pt}\large\sffamily\bfseries} % Spacing and font options for chapters
{\color{maincolor}\contentslabel[\Large\thecontentslabel]{1.25cm}\color{maincolor}} % Chapter number
{}
{\color{maincolor}\normalsize\sffamily\bfseries\;\titlerule*[.5pc]{.}\;\thecontentspage} % Page number

% Section text styling
\titlecontents{lsection}[0em] % Indenting
{\sffamily\small} % Spacing and font options for sections
{\contentslabel[\thecontentslabel]{1.25cm}} % Section number
{}
{}

% Subsection text styling
\titlecontents{lsubsection}[.5em] % Indentation
{\normalfont\footnotesize\sffamily} % Font settings
{}
{}
{}

%----------------------------------------------------------------------------------------
%	PAGE HEADERS
%----------------------------------------------------------------------------------------

\usepackage{fancyhdr} % Required for header and footer configuration

\pagestyle{fancy}
\renewcommand{\chaptermark}[1]{\markboth{\sffamily\normalsize\bfseries\chaptername\ \thechapter.\ #1}{}} % Chapter text font settings
\renewcommand{\sectionmark}[1]{\markright{\sffamily\normalsize\thesection\hspace{5pt}#1}{}} % Section text font settings
\fancyhf{} \fancyhead[LE,RO]{\sffamily\normalsize\thepage} % Font setting for the page number in the header
\fancyhead[LO]{\rightmark} % Print the nearest section name on the left side of odd pages
\fancyhead[RE]{\leftmark} % Print the current chapter name on the right side of even pages
\renewcommand{\headrulewidth}{0.5pt} % Width of the rule under the header
\addtolength{\headheight}{2.5pt} % Increase the spacing around the header slightly
\renewcommand{\footrulewidth}{0pt} % Removes the rule in the footer
\fancypagestyle{plain}{\fancyhead{}\renewcommand{\headrulewidth}{0pt}} % Style for when a plain pagestyle is specified

% Removes the header from odd empty pages at the end of chapters
\makeatletter
\renewcommand{\cleardoublepage}{
\clearpage\ifodd\c@page\else
\hbox{}
\vspace*{\fill}
\thispagestyle{empty}
\newpage
\fi}

%----------------------------------------------------------------------------------------
%	THEOREM STYLES
%----------------------------------------------------------------------------------------

\usepackage{amsmath,amsfonts,amssymb,amsthm} % For math equations, theorems, symbols, etc

\newcommand{\intoo}[2]{\mathopen{]}#1\,;#2\mathclose{[}}
\newcommand{\ud}{\mathop{\mathrm{{}d}}\mathopen{}}
\newcommand{\intff}[2]{\mathopen{[}#1\,;#2\mathclose{]}}
\newtheorem{notation}{Notation}[chapter]

% Boxed/framed environments
\newtheoremstyle{maincolornumbox}% % Theorem style name
{0pt}% Space above
{0pt}% Space below
{\normalfont}% % Body font
{}% Indent amount
{\small\bf\sffamily\color{maincolor}}% % Theorem head font
{\;}% Punctuation after theorem head
{0.25em}% Space after theorem head
{\small\sffamily\color{maincolor}\thmname{#1}\nobreakspace\thmnumber{\@ifnotempty{#1}{}\@upn{#2}}% Theorem text (e.g. Theorem 2.1)
\thmnote{\nobreakspace\the\thm@notefont\sffamily\bfseries\color{black}---\nobreakspace#3.}} % Optional theorem note
\renewcommand{\qedsymbol}{$\blacksquare$}% Optional qed square

\newtheoremstyle{blacknumex}% Theorem style name
{5pt}% Space above
{5pt}% Space below
{\normalfont}% Body font
{} % Indent amount
{\small\bf\sffamily}% Theorem head font
{\;}% Punctuation after theorem head
{0.25em}% Space after theorem head
{\small\sffamily{\tiny\ensuremath{\blacksquare}}\nobreakspace\thmname{#1}\nobreakspace\thmnumber{\@ifnotempty{#1}{}\@upn{#2}}% Theorem text (e.g. Theorem 2.1)
\thmnote{\nobreakspace\the\thm@notefont\sffamily\bfseries---\nobreakspace#3.}}% Optional theorem note

\newtheoremstyle{blacknumbox} % Theorem style name
{0pt}% Space above
{0pt}% Space below
{\normalfont}% Body font
{}% Indent amount
{\small\bf\sffamily}% Theorem head font
{\;}% Punctuation after theorem head
{0.25em}% Space after theorem head
{\small\sffamily\thmname{#1}\nobreakspace\thmnumber{\@ifnotempty{#1}{}\@upn{#2}}% Theorem text (e.g. Theorem 2.1)
\thmnote{\nobreakspace\the\thm@notefont\sffamily\bfseries---\nobreakspace#3.}}% Optional theorem note

% Non-boxed/non-framed environments
\newtheoremstyle{maincolornum}% % Theorem style name
{5pt}% Space above
{5pt}% Space below
{\normalfont}% % Body font
{}% Indent amount
{\small\bf\sffamily\color{maincolor}}% % Theorem head font
{\;}% Punctuation after theorem head
{0.25em}% Space after theorem head
{\small\sffamily\color{maincolor}\thmname{#1}\nobreakspace\thmnumber{\@ifnotempty{#1}{}\@upn{#2}}% Theorem text (e.g. Theorem 2.1)
\thmnote{\nobreakspace\the\thm@notefont\sffamily\bfseries\color{black}---\nobreakspace#3.}} % Optional theorem note
\renewcommand{\qedsymbol}{$\blacksquare$}% Optional qed square
\makeatother

% Defines the theorem text style for each type of theorem to one of the three styles above
\newcounter{dummy}
\numberwithin{dummy}{section}
\theoremstyle{maincolornumbox}
\newtheorem{theoremeT}[dummy]{Theorem}
\newtheorem{problem}{Problem}[chapter]
\newtheorem{exerciseT}{Exercise}[chapter]
\theoremstyle{blacknumex}
\newtheorem{exampleT}{Example}[chapter]
\theoremstyle{blacknumbox}
\newtheorem{vocabulary}{Vocabulary}[chapter]
\newtheorem{definitionT}{Definition}[section]
\newtheorem{corollaryT}[dummy]{Corollary}
\theoremstyle{maincolornum}
\newtheorem{proposition}[dummy]{Proposition}

%----------------------------------------------------------------------------------------
%	DEFINITION OF COLORED BOXES
%----------------------------------------------------------------------------------------

\RequirePackage[framemethod=default]{mdframed} % Required for creating the theorem, definition, exercise and corollary boxes

% Theorem box
\newmdenv[skipabove=7pt,
skipbelow=7pt,
backgroundcolor=black!5,
linecolor=maincolor,
innerleftmargin=5pt,
innerrightmargin=5pt,
innertopmargin=5pt,
leftmargin=0cm,
rightmargin=0cm,
innerbottommargin=5pt]{tBox}

% Exercise box
\newmdenv[skipabove=7pt,
skipbelow=7pt,
rightline=false,
leftline=true,
topline=false,
bottomline=false,
backgroundcolor=maincolor!10,
linecolor=maincolor,
innerleftmargin=5pt,
innerrightmargin=5pt,
innertopmargin=5pt,
innerbottommargin=5pt,
leftmargin=0cm,
rightmargin=0cm,
linewidth=4pt]{eBox}

% Definition box
\newmdenv[skipabove=7pt,
skipbelow=7pt,
rightline=false,
leftline=true,
topline=false,
bottomline=false,
linecolor=maincolor,
innerleftmargin=5pt,
innerrightmargin=5pt,
innertopmargin=0pt,
leftmargin=0cm,
rightmargin=0cm,
linewidth=4pt,
innerbottommargin=0pt]{dBox}

% Corollary box
\newmdenv[skipabove=7pt,
skipbelow=7pt,
rightline=false,
leftline=true,
topline=false,
bottomline=false,
linecolor=gray,
backgroundcolor=black!5,
innerleftmargin=5pt,
innerrightmargin=5pt,
innertopmargin=5pt,
leftmargin=0cm,
rightmargin=0cm,
linewidth=4pt,
innerbottommargin=5pt]{cBox}

% Creates an environment for each type of theorem and assigns it a theorem text style from the "Theorem Styles" section above and a colored box from above
\newenvironment{theorem}{\begin{tBox}\begin{theoremeT}}{\end{theoremeT}\end{tBox}}
\newenvironment{exercise}{\begin{eBox}\begin{exerciseT}}{\hfill{\color{maincolor}\tiny\ensuremath{\blacksquare}}\end{exerciseT}\end{eBox}}
\newenvironment{definition}{\begin{dBox}\begin{definitionT}}{\end{definitionT}\end{dBox}}
\newenvironment{example}{\begin{exampleT}}{\hfill{\tiny\ensuremath{\blacksquare}}\end{exampleT}}
\newenvironment{corollary}{\begin{cBox}\begin{corollaryT}}{\end{corollaryT}\end{cBox}}

%----------------------------------------------------------------------------------------
%	REMARK ENVIRONMENT
%----------------------------------------------------------------------------------------

\newenvironment{remark}{\par\vspace{10pt}\small % Vertical white space above the remark and smaller font size
\begin{list}{}{
\leftmargin=35pt % Indentation on the left
\rightmargin=25pt}\item\ignorespaces % Indentation on the right
\makebox[-2.5pt]{\begin{tikzpicture}[overlay]
\node[draw=maincolor!60,line width=1pt,circle,fill=maincolor!25,font=\sffamily\bfseries,inner sep=2pt,outer sep=0pt] at (-15pt,0pt){\textcolor{maincolor}{R}};\end{tikzpicture}} % Orange R in a circle
\advance\baselineskip -1pt}{\end{list}\vskip5pt} % Tighter line spacing and white space after remark

%----------------------------------------------------------------------------------------
%	SECTION NUMBERING IN THE MARGIN
%----------------------------------------------------------------------------------------

\makeatletter
\renewcommand{\@seccntformat}[1]{\llap{\textcolor{maincolor}{\csname the#1\endcsname}\hspace{1em}}}
\renewcommand{\section}{\@startsection{section}{1}{\z@}
{-4ex \@plus -1ex \@minus -.4ex}
{1ex \@plus.2ex }
{\normalfont\large\sffamily\bfseries}}
\renewcommand{\subsection}{\@startsection {subsection}{2}{\z@}
{-3ex \@plus -0.1ex \@minus -.4ex}
{0.5ex \@plus.2ex }
{\normalfont\sffamily\bfseries}}
\renewcommand{\subsubsection}{\@startsection {subsubsection}{3}{\z@}
{-2ex \@plus -0.1ex \@minus -.2ex}
{.2ex \@plus.2ex }
{\normalfont\small\sffamily\bfseries}}
\renewcommand\paragraph{\@startsection{paragraph}{4}{\z@}
{-2ex \@plus-.2ex \@minus .2ex}
{.1ex}
{\normalfont\small\sffamily\bfseries}}

%----------------------------------------------------------------------------------------
%	PART HEADINGS
%----------------------------------------------------------------------------------------

% numbered part in the table of contents
\newcommand{\@mypartnumtocformat}[2]{%
\setlength\fboxsep{0pt}%
\noindent\colorbox{maincolor!20}{\strut\parbox[c][.7cm]{\ecart}{\color{maincolor!70}\Large\sffamily\bfseries\centering#1}}\hskip\esp\colorbox{maincolor!40}{\strut\parbox[c][.7cm]{\linewidth-\ecart-\esp}{\Large\sffamily\centering#2}}}%
%%%%%%%%%%%%%%%%%%%%%%%%%%%%%%%%%%
% unnumbered part in the table of contents
\newcommand{\@myparttocformat}[1]{%
\setlength\fboxsep{0pt}%
\noindent\colorbox{maincolor!40}{\strut\parbox[c][.7cm]{\linewidth}{\Large\sffamily\centering#1}}}%
%%%%%%%%%%%%%%%%%%%%%%%%%%%%%%%%%%
\newlength\esp
\setlength\esp{4pt}
\newlength\ecart
\setlength\ecart{1.2cm-\esp}
\newcommand{\thepartimage}{}%
\newcommand{\partimage}[1]{\renewcommand{\thepartimage}{#1}}%
\def\@part[#1]#2{%
\ifnum \c@secnumdepth >-2\relax%
\refstepcounter{part}%
\addcontentsline{toc}{part}{\texorpdfstring{\protect\@mypartnumtocformat{\thepart}{#1}}{\partname~\thepart\ ---\ #1}}
\else%
\addcontentsline{toc}{part}{\texorpdfstring{\protect\@myparttocformat{#1}}{#1}}%
\fi%
\startcontents%
\markboth{}{}%
{\thispagestyle{empty}%
\begin{tikzpicture}[remember picture,overlay]%
\node at (current page.north west){\begin{tikzpicture}[remember picture,overlay]%
\fill[maincolor!20](0cm,0cm) rectangle (\paperwidth,-\paperheight);
\node[anchor=north] at (4cm,-3.25cm){\color{maincolor!40}\fontsize{220}{100}\sffamily\bfseries\@Roman\c@part};
\node[anchor=south east] at (\paperwidth-1cm,-\paperheight+1cm){\parbox[t][][t]{8.5cm}{
\printcontents{l}{0}{\setcounter{tocdepth}{1}}%
}};
\node[anchor=north east] at (\paperwidth-1.5cm,-3.25cm){\parbox[t][][t]{15cm}{\strut\raggedleft\color{white}\fontsize{30}{30}\sffamily\bfseries#2}};
\end{tikzpicture}};
\end{tikzpicture}}%
\@endpart}
\def\@spart#1{%
\startcontents%
\phantomsection
{\thispagestyle{empty}%
\begin{tikzpicture}[remember picture,overlay]%
\node at (current page.north west){\begin{tikzpicture}[remember picture,overlay]%
\fill[maincolor!20](0cm,0cm) rectangle (\paperwidth,-\paperheight);
\node[anchor=north east] at (\paperwidth-1.5cm,-3.25cm){\parbox[t][][t]{15cm}{\strut\raggedleft\color{white}\fontsize{30}{30}\sffamily\bfseries#1}};
\end{tikzpicture}};
\end{tikzpicture}}
\addcontentsline{toc}{part}{\texorpdfstring{%
\setlength\fboxsep{0pt}%
\noindent\protect\colorbox{maincolor!40}{\strut\protect\parbox[c][.7cm]{\linewidth}{\Large\sffamily\protect\centering #1\quad\mbox{}}}}{#1}}%
\@endpart}
\def\@endpart{\vfil\newpage
\if@twoside
\if@openright
\null
\thispagestyle{empty}%
\newpage
\fi
\fi
\if@tempswa
\twocolumn
\fi}

%----------------------------------------------------------------------------------------
%	CHAPTER HEADINGS
%----------------------------------------------------------------------------------------

% A switch to conditionally include a picture, implemented by  Christian Hupfer
\newif\ifusechapterimage
\usechapterimagetrue
\newcommand{\thechapterimage}{}%
\newcommand{\chapterimage}[1]{\ifusechapterimage\renewcommand{\thechapterimage}{#1}\fi}%
\def\@makechapterhead#1{%
{\parindent \z@ \raggedright \normalfont
\ifnum \c@secnumdepth >\m@ne
\if@mainmatter
\begin{tikzpicture}[remember picture,overlay]
\node at (current page.north west)
{\begin{tikzpicture}[remember picture,overlay]
\node[anchor=north west,inner sep=0pt] at (0,0) {\ifusechapterimage\includegraphics[width=\paperwidth]{\thechapterimage}\fi};
\draw[anchor=west] (\Gm@lmargin,-9cm) node [line width=2pt,rounded corners=15pt,draw=maincolor,fill=white,fill opacity=0.5,inner sep=15pt]{\strut\makebox[22cm]{}};
\draw[anchor=west] (\Gm@lmargin+.3cm,-9cm) node {\huge\sffamily\bfseries\color{black}\thechapter. #1\strut};
\end{tikzpicture}};
\end{tikzpicture}
\else
\begin{tikzpicture}[remember picture,overlay]
\node at (current page.north west)
{\begin{tikzpicture}[remember picture,overlay]
\node[anchor=north west,inner sep=0pt] at (0,0) {\ifusechapterimage\includegraphics[width=\paperwidth]{\thechapterimage}\fi};
\draw[anchor=west] (\Gm@lmargin,-9cm) node [line width=2pt,rounded corners=15pt,draw=maincolor,fill=white,fill opacity=0.5,inner sep=15pt]{\strut\makebox[22cm]{}};
\draw[anchor=west] (\Gm@lmargin+.3cm,-9cm) node {\huge\sffamily\bfseries\color{black}#1\strut};
\end{tikzpicture}};
\end{tikzpicture}
\fi\fi\par\vspace*{270\p@}}}

%-------------------------------------------

\def\@makeschapterhead#1{%
\begin{tikzpicture}[remember picture,overlay]
\node at (current page.north west)
{\begin{tikzpicture}[remember picture,overlay]
\node[anchor=north west,inner sep=0pt] at (0,0) {\ifusechapterimage\includegraphics[width=\paperwidth]{\thechapterimage}\fi};
\draw[anchor=west] (\Gm@lmargin,-9cm) node [line width=2pt,rounded corners=15pt,draw=maincolor,fill=white,fill opacity=0.5,inner sep=15pt]{\strut\makebox[22cm]{}};
\draw[anchor=west] (\Gm@lmargin+.3cm,-9cm) node {\huge\sffamily\bfseries\color{black}#1\strut};
\end{tikzpicture}};
\end{tikzpicture}
\par\vspace*{270\p@}}
\makeatother

%----------------------------------------------------------------------------------------
%	HYPERLINKS IN THE DOCUMENTS
%----------------------------------------------------------------------------------------

\usepackage{hyperref}
\hypersetup{hidelinks,backref=true,pagebackref=true,hyperindex=true,colorlinks=false,breaklinks=true,urlcolor= maincolor,bookmarks=true,bookmarksopen=false,pdftitle={Title},pdfauthor={Author}}
\usepackage{bookmark}
\bookmarksetup{
open,
numbered,
addtohook={%
\ifnum\bookmarkget{level}=0 % chapter
\bookmarksetup{bold}%
\fi
\ifnum\bookmarkget{level}=-1 % part
\bookmarksetup{color=maincolor,bold}%
\fi
}
}

%----------------------------------------------------------------------------------------
%	Java source code
%----------------------------------------------------------------------------------------

% Commando voor invoegen Java-broncodebestanden (dank aan Niels Corneille)
% Gebruik:
%   \codefragment{source/MijnKlasse.java}{Uitleg bij de code}
%
% Je kan dit aanpassen aan de taal die je zelf het meeste gebruikt in je
% bachelorproef.
\newcommand{\codefragment}[2]{ \lstset{%
  language=bash,
  breaklines=true,
  float=th,
  caption={#2},
  basicstyle=\scriptsize,
  frame=single,
  extendedchars=\true
}
\lstinputlisting{#1}}

% Leeg blad
\newcommand{\emptypage}{%
\newpage
\thispagestyle{empty}
\mbox{}
\newpage
}


%%---------- Documenteigenschappen --------------------------------------------
%% TODO: Vul dit aan met je eigen info:

% Je eigen naam
\newcommand{\student}{Jonas De Moor}

% De naam van je promotor (lector van de opleiding)
\newcommand{\promotor}{Antonia Pierreux}

% De naam van je co-promotor. Als je promotor ook je opdrachtgever is en je
% dus ook inhoudelijk begeleidt (en enkel dan!), mag je dit leeg laten.
\newcommand{\copromotor}{Karine Van Driessche}

% Indien je bachelorproef in opdracht van/in samenwerking met een bedrijf of
% externe organisatie geschreven is, geef je hier de naam. Zoniet laat je dit
% zoals het is.
\newcommand{\instelling}{---}

% De titel van het rapport/bachelorproef
\newcommand{\titel}{ZFS met RAID-Z als alternatief voor klassieke RAID-oplossingen}

% Datum van indienen (gebruik telkens de deadline, ook al geef je eerder af)
\newcommand{\datum}{2 juni 2017}

% Academiejaar
\newcommand{\academiejaar}{2016-2017}

% Examenperiode
%  - 1e semester = 1e examenperiode => 1
%  - 2e semester = 2e examenperiode => 2
%  - tweede zit  = 3e examenperiode => 3
\newcommand{\examenperiode}{2}

%%=============================================================================
%% Inhoud document
%%=============================================================================

\begin{document}

%---------- Taalselectie ------------------------------------------------------
%% Als je je bachelorproef in het Engels schrijft, haal dan onderstaande regel
%% uit commentaar. Let op: de tekst op de voorkaft blijft in het Nederlands, en
%% dat is ook de bedoeling!
%\selectlanguage{english}

%---------- Titelblad ---------------------------------------------------------
\inserttitlepage

%---------- Samenvatting, voorwoord -------------------------------------------
\usechapterimagefalse
%%=============================================================================
%% Samenvatting
%%=============================================================================

%% TODO: De "abstract" of samenvatting is een kernachtige (~ 1 blz. voor een
%% thesis) synthese van het document.
%%
%% Deze aspecten moeten zeker aan bod komen:
%% - Context: waarom is dit werk belangrijk?
%% - Nood: waarom moest dit onderzocht worden?
%% - Taak: wat heb je precies gedaan?
%% - Object: wat staat in dit document geschreven?
%% - Resultaat: wat was het resultaat?
%% - Conclusie: wat is/zijn de belangrijkste conclusie(s)?
%% - Perspectief: blijven er nog vragen open die in de toekomst nog kunnen
%%    onderzocht worden? Wat is een mogelijk vervolg voor jouw onderzoek?
%%
%% LET OP! Een samenvatting is GEEN voorwoord!

%%---------- Nederlandse samenvatting -----------------------------------------
%%
%% TODO: Als je je bachelorproef in het Engels schrijft, moet je eerst een
%% Nederlandse samenvatting invoegen. Haal daarvoor onderstaande code uit
%% commentaar.
%% Wie zijn bachelorproef in het Nederlands schrijft, kan dit negeren en heel
%% deze sectie verwijderen.

%\IfLanguageName{english}{%
%\selectlanguage{dutch}
%\chapter*{Samenvatting}
%\lipsum[1-4]
%\selectlanguage{english}
%}{}

%%---------- Samenvatting -----------------------------------------------------
%%
%% De samenvatting in de hoofdtaal van het document

\chapter*{\IfLanguageName{dutch}{Samenvatting}{Abstract}}

%\lipsum[1-4]

RAID (Redundant Array of Independent Disks) is een technologie die al lange tijd is ingeburgerd in bedrijven. Systeembeheerders gebruiken RAID vooral om ervoor te zorgen dat één of meerdere defecte schijven niet kunnen leiden tot dataverlies, maar dit hoeft niet noodzakelijk het geval te zijn. 

In het laatste decennium zijn softwaregebaseerde RAID-oplossingen steeds populairder geworden. Eén van deze oplossingen is RAID-Z, een softwarematige RAID die deel uitmaakt van de ZFS storage stack. ZFS is een geavanceerd bestandssysteem ontwikkeld door het vroegere Sun Microsystems in het begin van de jaren 2000. 

In deze bachelorproef wordt achterhaald of ZFS en RAID-Z een goed alternatief zouden vormen voor een meer traditionele RAID-oplossingen, zoals een hardwaregebaseerde RAID. De motivatie voor het voeren van een onderzoek naar een alternatieven voor RAID is hoofdzakelijk omdat traditionele RAID5 arrays nog steeds onderhevig zijn aan het zgn. "RAID5 write hole", waarbij dataverlies kan optreden bij bijvoorbeeld een stroompanne. Ook bieden de meeste RAID-controllers geen bescherming tegen silent data corruption.

De scriptie is opgedeeld in twee grote delen: een theoretisch deel en een praktisch deel. In het theoretische deel worden de basisprincipes van RAID en de architectuur van ZFS in vogelvlucht overlopen. Het praktische gedeelte behandelt voornamelijk de performantie en betrouwbaarheid van ZFS en RAID-Z. 
De performantietesten en het grootste deel van het praktische deel werd uitgevoerd m.b.v. een HP desktopsysteem; de betrouwbaarheidstesten werden uitgevoerd met een virtuele machine via VirtualBox. Hierbij werd geconstateerd dat ZFS zijn beloftes m.b.t. betrouwbaarheid waarmaakt en de data goed beschermt tegen datacorruptie en hardwarefalen. Performantie van ZFS is uitstekend te noemen; de performantie werd vergeleken met die van Linux MD, een softwarematige RAID voor Linux, en werd getest m.b.v. de Phoronix Test Suite.

Afhankelijk van de use case (situatie, nodige opslagcapaciteit, beschikbare hardware) vormt ZFS met RAID-Z in het merendeel van de gevallen een uitstekend alternatief voor een klassieke RAID-oplossing. In de toekomst zullen ZFS en andere COW-bestandssystemen, zoals APFS, ReFS en BTRFS, naar alle waarschijnlijkheid interessanter worden voor dagelijks gebruik; een vergelijkende studie tussen deze verschillende bestandssystemen zou nog een interessante toevoeging zijn aan deze scriptie.

%%=============================================================================
%% Voorwoord
%%=============================================================================

\chapter*{Voorwoord}
\label{ch:voorwoord}

%% TODO:
%% Het voorwoord is het enige deel van de bachelorproef waar je vanuit je
%% eigen standpunt (``ik-vorm'') mag schrijven. Je kan hier bv. motiveren
%% waarom jij het onderwerp wil bespreken.
%% Vergeet ook niet te bedanken wie je geholpen/gesteund/... heeft

Deze bachelorproef duidt het einde aan van mijn opleiding Toegepaste Informatica. Met deze bachelorproef wil ik bewijzen dat ik op een zelfstandige en objectieve manier onderzoek kan voeren over een (nieuwe) technologie in de IT-wereld. Omdat onze branche vrijwel continu onderhevig is aan verandering, vind ik dit een niet-onbelangrijke competentie. 

Als onderwerp van deze bachelorproef heb ik besloten om een al wat oudere (maar zeker niet oninteressante) technologie te bespreken: het ZFS bestandssysteem. De redenen waarom ik net dit onderwerp zou willen bespreken, zijn nogal uiteenlopend. Ik ben zelf een enorme Linux- en Unix-fan en ik hou ervan om mezelf nieuwe dingen aan te leren. Met ZFS was ik nog niet vertrouwd, en daar ik toch van plan ben om zelf een homeserver samen te stellen en als bestandssysteem ZFS te gebruiken, leek deze bachelorproef mij een uitstekende opportuniteit om mij wat meer te verdiepen in de werking en implementatie van deze technoogie. Ook hoorde ik hier en daar geruchten vallen over de mogelijke onbetrouwbaarheid van RAID (het zgn.RAID 5 "write hole") en dit was dan ook een reden om onderzoek te voeren naar een mogelijk alternatief.

Deze bachelorproef is het resultaat van vele uren noeste arbeid; het is een werk waar ik enorm trots op ben. Desalniettemin zou dit werk niet mogelijk zijn geweest zonder de hulp van een aantal mensen. Graag neem ik daarom even de tijd om enkele personen te bedanken voor hun steun en toeverlaat gedurende deze periode. 

Eerst en vooral zou ik mijn promotor, mevr. Antonia Pierreux, en mijn co-promotor, mevr. Karine Van Driessche, willen bedanken voor het goed laten verlopen van deze periode. Het schrijven van deze scriptie en het voeren van een onderzoek waren geen makkelijke karweien; zonder hun inhoudelijke en technische steun zou deze bachelorproef nooit mogelijk geweest zijn. Tevens zou ik ook nog graag mijn familie en vrienden willen bedanken voor de onophoudelijke steun en begrip. Ook wil ik graag de mensen van DViT bedanken voor het aanreiken van technische kennis en materiaal voor mijn onderzoek. En \textit{last but not least} wil ik mijn ouders enorm bedanken om mij gedurende deze drie jaar ten volle te steunen: dankzij hen heb ik telkens opnieuw de moed teruggevonden om er met volle teugen tegenaan te gaan in perioden dat het wat moeilijker ging.

Ik hoop dat u evenveel plezier beleeft met het lezen van mijn scriptie als ik had met het schrijven ervan.

\begin{flushright}
  \textit{Jonas De Moor}
  \textit{Academiejaar 2016-2017}
\end{flushright}


%---------- Inhoudstafel ------------------------------------------------------
\pagestyle{empty} % No headers
\tableofcontents % Print the table of contents itself
\cleardoublepage % Forces the first chapter to start on an odd page so it's on the right
\pagestyle{fancy} % Print headers again

%---------- Lijst afkortingen, termen -----------------------------------------
%% Als je een lijst van afkortingen of termen wil toevoegen, dan hoort die
%% hier thuis. Gebruik bijvoorbeeld de ``glossaries'' package.

%%---------- Kern -------------------------------------------------------------

%%=============================================================================
%% Inleiding
%%=============================================================================

\chapter{Inleiding}
\label{ch:inleiding}

Al jarenlang is de meest gebruikte oplossing tegen dataverlies door het falen van schijven  RAID, wat staat voor Redundant Array of Independent (of Inexpensive) Disks. RAID is echter niet geheel feilloos: het beschermt bv. niet tegen fouten die gemaakt zijn door gebruikers (denk maar aan het wissen van belangrijke data) en het biedt ook geen oplossing als er in een zelfde tijdsbestek meerdere schijven falen \autocite{Chen1994}.

In 2002 begon het toenmalige Sun Microsystems, nu onderdeel van Oracle Corporation, aan de ontwikkeling van ZFS (Zettabyte Filesystem). Dit is een bestandssysteem dat volledig \textit{from scratch} is ontwikkeld om \textit{``de tekortkomingen van huidige bestandssystemen op te lossen''}\autocite{JeffBonwick_lastZFS}, vooral met betrekking tot data-integriteit. Volgens de toenmalige hoofdontwikkelaar \textcite{ZFSBonwick} biedt ZFS heel wat vernieuwende functionaliteiten zoals een eenvoudiger beheer, automatische foutcorrectie, automatisch schalende bestandssystemen en een softwarematige RAID onder de term \textit{RAID-Z}. 

Sinds 2013 is ZFS ook beschikbaar voor Linux, maar de eerste uitgaven leden onder nogal wat stabiliteitsproblemen. Tegenwoordig is ZFS op Linux volwassen genoeg geworden om in te zetten in productie. Deze bachelorproef is dan ook een goede gelegenheid om zelf wat meer onderzoek te doen naar ZFS en of het een volwaardig alternatief zou zijn voor een klassieke, hardwaregebaseerde RAID-oplossing op een server.

\section{Probleemstelling en Onderzoeksvragen}
\label{sec:onderzoeksvragen}

%% TODO:
%% Uit je probleemstelling moet duidelijk zijn dat je onderzoek een meerwaarde
%% heeft voor een concrete doelgroep (bv. een bedrijf).
%%
%% Wees zo concreet mogelijk bij het formuleren van je
%% onderzoeksvra(a)g(en). Een onderzoeksvraag is trouwens iets waar nog
%% niemand op dit moment een antwoord heeft (voor zover je kan nagaan).

Aangezien (een deel van) de RAID-functionaliteit van BTRFS nog niet als \textit{production-ready} wordt beschouwd \autocite{Project2017}, zal ZFS worden beschouwd en geëvalueerd als volwaardig alternatief voor klassieke RAID-oplossingen op Linux-systemen. 

Deze bachelorproef zal een antwoord vinden op volgende vragen:

\begin{itemize}
	\item{Wat zijn de grootste verschillen tussen een klassieke RAID-oplossing en ZFS RAID-Z?}
	\item{Hoe is de architectuur van ZFS opgebouwd en op welke manieren tracht het oplossingen te vinden voor de problemen die zich voordoen bij andere bestandssystemen en RAID-opstellingen?}
  \item{Hoe staat het met data-integriteit en \gls{performantie}\footnote{Met 'performantie' wordt het aantal I/O's per seconde en de globale schijf- en CPU-belasting bedoeld.} bij ZFS onder verschillende workloads en toepassingen?}
\end{itemize}

\section{Opzet van deze bachelorproef}
\label{sec:opzet-bachelorproef}

%% TODO: Het is gebruikelijk aan het einde van de inleiding een overzicht te
%% geven van de opbouw van de rest van de tekst. Deze sectie bevat al een aanzet
%% die je kan aanvullen/aanpassen in functie van je eigen tekst.

De rest van deze bachelorproef is als volgt opgebouwd:

In Hoofdstuk~\ref{ch:methodologie} wordt de methodologie toegelicht en worden de gebruikte onderzoekstechnieken besproken om een antwoord te formuleren op de onderzoeksvragen.

In Hoofdstuk~\ref{ch:h2} wordt er een korte inleiding gegeven op de geschiedenis en de algemene werking van RAID-systemen. Ook ZFS en RAID-Z worden reeds kort toegelicht.

In Hoofdstuk~\ref{ch:h3} wordt er een globaal overzicht gegeven van de architectuur en ontwerpprincipes van ZFS. In de daaropvolgende hoofdstukken worden de belangrijkste onderdelen en functionaliteiten wat meer uitgediept. 

In Hoofdstuk \ref{ch:h4} wordt het opslagmodel van het ZFS-bestandssysteem besproken. Onder andere de datastructuur en het transactiemodel van ZFS komen aan bod.

In Hoofdstuk \ref{ch:h5} worden de stappen die moeten worden ondernomen om een desktopcomputer om te zetten naar een Linux-server die kan worden gebruikt voor ZFS besproken.

In Hoofdstuk \ref{ch:h6} worden zpools en VDEV's wat meer in detail belicht. Tevens wordt er gedemonstreerd hoe men zpools en VDEV's aanmaakt en wijzigt.


In Hoofdstuk~\ref{ch:conclusie}, tenslotte, wordt de conclusie gegeven en een antwoord geformuleerd op de onderzoeksvragen. Daarbij wordt ook een aanzet gegeven voor toekomstig onderzoek binnen dit domein.


%%=============================================================================
%% Methodologie
%%=============================================================================

\chapter{Methodologie}
\label{ch:methodologie}

%% TODO: Hoe ben je te werk gegaan? Verdeel je onderzoek in grote fasen, en
%% licht in elke fase toe welke stappen je gevolgd hebt. Verantwoord waarom je
%% op deze manier te werk gegaan bent. Je moet kunnen aantonen dat je de best
%% mogelijke manier toegepast hebt om een antwoord te vinden op de
%% onderzoeksvraag.

In dit hoofdstuk worden de methodes en denkpistes besproken die werden gehanteerd tijdens het opstellen van deze scriptie. Daarnaast wordt er reeds per hoofdstuk een inhoudelijk overzicht gegeven van wat de lezer kan verwachten bij het lezen van dit werk.

\section{Gehanteerde methodiek}

De bachelorproef is opgedeeld in twee grote onderdelen: een theoretisch deel en een meer praktisch gericht deel. 

In het theoretische gedeelte wordt er vooral gefocust op de interne werking van ZFS. Vooraleer echter de werking van ZFS uit te spitten, wordt er eerst een overzicht gegeven van RAID-systemen en RAID-niveaus. Nadien worden het ontwerp van ZFS en de beslissingen van de ontwikkelaars toegelicht; waar mogelijk wordt er telkens een vergelijking gemaakt met de manier waarop meer 'traditionele' oplossingen een bepaald probleem zouden aanpakken. Niet alle aspecten van de interne werking van ZFS worden besproken; daarvoor is deze scriptie ook helemaal niet bedoeld. Echter is een globaal beeld van de werking van ZFS van belang aangezien er toch wel significante verschillen zijn tussen een opslagstack binnen ZFS en een traditionele opslagstack. 

Het theoretisch deel biedt m.a.w. al grotendeels  een antwoord op de eerste twee onderzoeksvragen.

In het praktische gedeelte worden onder andere de performantie en betrouwbaarheid van ZFS geanalyseerd, om zo een antwoord te vinden op de laatste onderzoeksvraag. Bij dit onderdeel worden er twee testsystemen gebruikt, nl. een fysieke machine en een virtuele machine (via VirtualBox). Het eerstegenoemde systeem dient hoofdzakelijk om benchmarks uit te voeren; het tweede systeem dient uitsluitend om de betrouwbaarheid van een ZFS RAID-Z-opstelling na te gaan.

Voor het uitvoeren van de performantietesten wordt er gebruik gemaakt van \textbf{Phoronix Test Suite\footnote{\url{https://www.phoronix-test-suite.com}}}: deze suite is een wrapper rond veelgebruike benchmarktools en maakt het mogelijk om op een makkelijke manier relevante gegevens te verzamelen. Er is weinig tot geen voorafgaande kennis vereist voor het uitvoeren van de verschillende benchmarks, en dit was dan ook één van de hoofdredenen om voor dit programma te kiezen. 

Naast performantie en betrouwbaarheid, worden er ook nog andere aspecten van ZFS belicht, waaronder:

\begin{itemize}
  \item{Het voorbereiden en installeren van een computersysteem voor het gebruik van Linux en ZFS;}
  \item{Creatie en beheer van ZFS pools en VDEV's;}
  \item{De verschillende types van bestandssystemen (of datasets) die er binnen de ZFS stack bestaan;}
\end{itemize}

Hier en daar worden er nog enkele theoretische aspecten besproken, maar enkel al alleen als dit een toegevoegde waarde heeft. Bij bijvoorbeeld het hoofdstuk over VDEV's en storage pools is het noodzakelijk om te verduidelijken welke soorten VDEV's er bestaan; op deze manier wordt er context geschapen en is het voor de lezer ook duidelijker wat er in bepaalde gevallen bedoeld wordt.

\section{Opbouw van de bachelorproef}

Deze bachelorproef is verder globaal gezien als volgt opgebouwd:

In Hoofdstuk \ref{ch:methodologie} wordt de methodologie toegelicht en worden de gebruikte onderzoekstech-
nieken besproken om een antwoord te formuleren op de onderzoeksvragen.

In Hoofdstuk \ref{ch:h2} wordt er een korte inleiding gegeven op de geschiedenis en de algemene
werking van RAID-systemen. Ook ZFS en RAID-Z worden reeds kort toegelicht.

In Hoofdstuk \ref{ch:h3} wordt er een globaal overzicht gegeven van de architectuur en ontwerpprin-
cipes van ZFS. In de daaropvolgende hoofdstukken worden de belangrijkste onderdelen en
functionaliteiten wat meer uitgediept.

In Hoofdstuk \ref{ch:h4} wordt het opslagmodel van het ZFS-bestandssysteem besproken. Onder
andere de datastructuur en het transactiemodel van ZFS komen aan bod.

In Hoofdstuk \ref{ch:h5} worden de stappen die moeten worden ondernomen om een desktopcom-
puter om te zetten naar een Linux-server die kan worden gebruikt voor ZFS besproken.

In Hoofdstuk \ref{ch:h6} worden zpools en VDEV’s wat meer in detail belicht. Tevens wordt er
gedemonstreerd hoe men zpools en VDEV’s aanmaakt en wijzigt.

In Hoofdstuk \ref{ch:h7} worden traditionele bestandssystemen vergeleken met ZFS datasets. Onder andere de verschillende soorten datasets komen aan bod; tevens wordt er aan het eind van het hoofdstuk getoond hoe een ZFS dataset kan worden gebruikt om een NFS-share op te zetten.

In Hoofdstuk \ref{ch:h8} worden de prestaties van RAID-Z en Linux MD, een softwarematige RAID binnen Linux, met elkaar vergeleken. Hiervoor wordt er gebruik gemaakt van Phoronix Benchmark: dit is een wrapper rond verschillende onafhankelijke tools dat het verzamelen van relevante gegevens een stuk makkelijker maakt.

In Hoofdstuk \ref{ch:h9} wordt de betrouwbaarheid van ZFS nagegaan, met name: hoe regaeert een RAID-Z-opstelling op fouten onder verschillende omstandigheden?

In Hoofdstuk \ref{ch:conclusie}, tenslotte, wordt de conclusie gegeven en een antwoord geformuleerd op
de onderzoeksvragen. Daarbij wordt ook een aanzet gegeven voor toekomstig onderzoek
binnen dit domein.


%% Voeg hier je eigen hoofdstukken toe die de ``corpus'' van je bachelorproef
%% vormen. De structuur en titels hangen af van je eigen onderzoek. Je kan bv.
%% elke fase in je onderzoek in een apart hoofdstuk bespreken.

%%=============================================================================
%% H3 - Architectuur en ontwerpprincipes van ZFS
%%=============================================================================

\chapter{Architectuur en ontwerpprincipes van ZFS}
\label{ch:h3}

\section{Ontwerpprincipes}

%\input{}
%...

%%=============================================================================
%% Conclusie
%%=============================================================================

\chapter{Conclusie}
\label{ch:conclusie}

%% TODO: Trek een duidelijke conclusie, in de vorm van een antwoord op de
%% onderzoeksvra(a)g(en). Wat was jouw bijdrage aan het onderzoeksdomein en
%% hoe biedt dit meerwaarde aan het vakgebied/doelgroep? Reflecteer kritisch
%% over het resultaat. Had je deze uitkomst verwacht? Zijn er zaken die nog
%% niet duidelijk zijn? Heeft het ondezoek geleid tot nieuwe vragen die
%% uitnodigen tot verder onderzoek?

%\lipsum[76-80]

Doorheen deze bachelorproef is het (hopelijk) duidelijk geworden dat het gebruik van de term 'bestandssysteem' ZFS eigenlijk een beetje onrecht aandoet. De afkorting 'ZFS' mag dan wel Zettabyte Filesystem betekenen, toch is ZFS veel meer dan een bestandssysteem alleen: het is een volledig nieuwe implementatie van de traditionele opslagstack zoals we die al jaren gewend zijn. Daar waar meer traditionele oplossingen vaak bestaan uit verschillende, losse onderdelen die met elkaar samenwerken, bestaat de ZFS storage stack uit een aantal nauw samenwerkende lagen. Toch voelt dit niet aan als een groot, log geheel; functionaliteiten zijn duidelijk afgebakend en dit weerspiegelt zich ook in het gebruik en beheer.   

Toen ZFS voor het eerst op het toneel verscheen, waren vele techneuten laaiend enthousiast: problemen zoals datacorruptie en inflexibele opslag zouden eindelijk van de baan zijn.  


%%---------- Back matter ------------------------------------------------------

\printbibliography
\addcontentsline{toc}{chapter}{\textcolor{maincolor}{\IfLanguageName{dutch}{Bibliografie}{Bibliography}}}


\listoffigures
\listoftables

\end{document}
