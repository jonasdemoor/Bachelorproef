%%=============================================================================
%% Inleiding
%%=============================================================================

\chapter{Inleiding}
\label{ch:inleiding}

%De inleiding moet de lezer alle nodige informatie verschaffen om het onderwerp te begrijpen zonder nog externe werken te moeten raadplegen \autocite{Pollefliet2011}. Dit is een doorlopende tekst die gebaseerd is op al wat je over het onderwerp gelezen hebt (literatuuronderzoek).

%Je verwijst bij elke bewering die je doet, vakterm die je introduceert, enz. naar je bronnen. In \LaTeX{} kan dat met het commando \texttt{$\backslash${textcite\{\}}} of \texttt{$\backslash${autocite\{\}}}. Als argument van het commando geef je de ``sleutel'' van een ``record'' in een bibliografische databank in het Bib\TeX{}-formaat (een tekstbestand). Als je expliciet naar de auteur verwijst in de zin, gebruik je \texttt{$\backslash${}textcite\{\}}.
%Soms wil je de auteur niet expliciet vernoemen, dan gebruik je \texttt{$\backslash${}autocite\{\}}. Hieronder een voorbeeld van elk.

%\textcite{Knuth1998} schreef een van de standaardwerken over sorteer- en zoekalgoritmen. Experten zijn het erover eens dat cloud computing een interessante opportuniteit vormen, zowel voor gebruikers als voor dienstverleners op vlak van informatietechnologie~\autocite{Creeger2009}.

%\section{Stand van zaken}
%\label{sec:stand-van-zaken}

%% TODO: deze sectie (die je kan opsplitsen in verschillende secties) bevat je
%% literatuurstudie. Vergeet niet telkens je bronnen te vermelden!

%\lipsum[7-20]

In dit hoofdstuk wordt er een korte inleiding gegeven over de basisprincipes van RAID, aangezien RAID-Z een softwarematige vorm van RAID is. Daarnaast wordt de geschiedenis en globale werking van ZFS reeds kort besproken. Aan het einde van dit hoofdstuk kunnen de probleemstelling en onderzoeksvragen worden teruggevonden, samen met de verdere indeling van deze bachelorproef.

\section{RAID}

Al van oudscher worden magnetische harde schijven gebruikt als opslagmedium voor data \autocite{Goda2012}. Maar reeds in de jaren 80 zagen onderzoekers in dat I/O-performance een bottleneck zou vormen voor computersystemen in de toekomst. Terwijl geheugenchips en processoren steeds sneller werden, bevonden opslagmedia zich in een impasse \autocite{DavidA.Paterson1987}.

In de paper \textit{"A Case for Redundant Arrays of Inexpensive Disks"}  formuleerden \textcite{DavidA.Paterson1987} en zijn collega's voor het eerst de term 'RAID', wat een acroniem is voor 'Redundant Arrays of Inexpensive Disks'. De oorspronkelijke idee achter RAID was dat een verzameling van goedkopere schijven performanter zou zijn dan grotere en duurdere mainframeschijven van die tijd. 

Naast performantie en kost was betrouwbaarheid (reliability)  ook een belangrijke factor. Als bv. één of meerdere schijven van de array falen, dan mag dit geen invloed hebben op de werking van de rest van de verzameling schijven. Daarom introduceerden de onderzoekers de zgn. "RAID levels"    \autocite{DavidA.Paterson1987}, die vandaag de dag nog steeds in gebruik zijn. Er bestaan een aantal RAID-niveaus, waarvan de voornaamste zullen besproken worden.

Bij het bouwen van RAID-systemen worden er gebruikelijk drie aspecten in beschouwing genomen: \textbf{performantie} (performance), \textbf{betrouwbaarheid} (reliability) en \textbf{capaciteit} (capacity). Een RAID-niveau is in principe niets anders dan een balans tussen deze verschillende eigenschappen; meestal zullen er dus één of meerdere trade-offs moeten gemaakt worden \autocite{Chen1994}.

Begrippen die centraal staan bij RAID zijn \textit{\textbf{striping}} en \textit{\textbf{parity}}. Striping heeft betrekking op de manier waarop een RAID-controller (hetzij hardwarematig, hetzij softwarematig) de blokken data verdeelt over de array van schijven. Bij RAID 0 bijvoorbeeld wordt de data gelijkmatig gedistribueerd volgens het \textit{round-robin}-algoritme \autocite{OSThreePiecesRemzi2015}. Datablokken die verdeeld zijn over meerdere schijven en samen één geheel vormen worden een \textit{stripe} genoemd. \\ 
Een voordeel bij RAID 0 is dat de gehele capaciteit van de schijven kan gebruikt worden; er gaat geen ruimte verloren, aangezien de data gelijkmatig verdeeld wordt over de array. Een bijkomend voordeel van striping is dat performantie in het algemeen goed is: de meeste en reads en writes kunnen parallel worden afgehandeld. Een voorwaarde voor goede performantie is echter wel dat de \textit{chunk size} (de grootte van de blokken data die worden weggeschreven en/of uitgelezen) ook optimaal wordt gekozen, i.e. afhankelijk van de workload op het systeem \autocite{OSThreePiecesRemzi2015}. Toch kent RAID 0 een groot nadeel: betrouwbaarheid is nagenoeg onbestaande. Aangezien data nergens wordt gedupliceerd, betekent dat het falen van eender welke disk leidt tot verlies van data \autocite{OSThreePiecesRemzi2015}.

Parity is een mechanisme dat werd geïntroduceerd bij RAID-niveau 4 om betrouwbaarheid af te dwingen. Bij RAID 4 wordt er metadata over de opgeslagen data bijgehouden in parity blocks op een aparte schijf. Deze metadata wordt verkregen d.m.v. het uitvoeren van een mathematische functie op de opgeslagen data. Meestal is dit een XOR-functie (exclusieve OF) \autocite{Chen1994}. Aan de hand van deze parity kan bij het verlies van één of meerdere schijven de originele data worden gereconstrueerd door XOR'ing toe te passen op de parity bits en de data bits. Bij een XOR-operatie geven een even aantal enen (1) steeds als resultaat nul (0); omgekeerd geldt ook dat een oneven aantal enen (1) steeds een één (1) als resultaat zullen opleveren. Stel dat één schijf van een array van vier schijven faalt, dan kan nog steeds de originele data worden verkregen. Echter, als er meer dan één schijf verloren gaat, dan is het bij RAID 4 onmogelijk om de originele data te herstellen. Het grote voordeel van RAID 4 is dan weer echter dat er minder wordt ingeboet op capaciteit  dan bij bv. RAID 1 en RAID 5 \autocite{OSThreePiecesRemzi2015}. 

Naast RAID 0 en RAID 4 zijn er nog andere RAID-levels, zoals RAID 1 en RAID 5. RAID 1 staat ook bekend als \textit{mirroring}, omdat het kopieën maakt van de datablokken naar één of meerdere disks afhankelijk van het aantal schijven. Op gebied van capaciteit is RAID 1 niet echt gunstig, aangezien maar de helft van de totale schijfruimte bruikbaar is. Stel dat er vier schijven in een array aanwezig zijn, dan is slechts de opslagcapaciteit van twee schijven bruikbaar. Daarentegen is de betrouwbaarheid van RAID 1 wel vele malen beter dan die van RAID 0: in theorie mogen er bij een reeks van \textit{n} schijven $\frac{n}{2}$ schijven falen. Maar dan mogen de schijven die elkaars mirror zijn niet falen, want dan is de data op deze disks verloren.  Daarom houdt men in de praktijk meestal de maatstaf van één schijf aan \autocite{OSThreePiecesRemzi2015}.

Als laatste wordt RAID 5 besproken. RAID 5 is in principe niets anders dan RAID-niveau 4, maar dan uitgebreid met functionaliteit dat de parity blocks roteert over de verschillende schijven. Dit is een groot verschil t.o.v. RAID 4, waarbij de parity blocks zich op één disk bevinden. Read-performantie is nagenoeg gelijk aan RAID 4, maar write-performantie is stukken beter. Dit komt omdat bij RAID 5 de schrijfoperaties parallel kunnen worden afgehandeld; bij RAID 4 vormt de parity-schijf een \textit{bottleneck} bij het wegschrijven van data. De reden hiervoor is dat bij het updaten van data ook de parity blocks moeten worden geüpdatet; alle operaties worden dus m.a.w. serieel uitgevoerd \autocite{OSThreePiecesRemzi2015}.

Er bestaan nog andere, niet-standaard RAID levels, zoals RAID 6 en RAID 10, maar deze worden hier niet besproken.

\section{ZFS}



\section{Probleemstelling en Onderzoeksvragen}
\label{sec:onderzoeksvragen}

%% TODO:
%% Uit je probleemstelling moet duidelijk zijn dat je onderzoek een meerwaarde
%% heeft voor een concrete doelgroep (bv. een bedrijf).
%%
%% Wees zo concreet mogelijk bij het formuleren van je
%% onderzoeksvra(a)g(en). Een onderzoeksvraag is trouwens iets waar nog
%% niemand op dit moment een antwoord heeft (voor zover je kan nagaan).

\section{Opzet van deze bachelorproef}
\label{sec:opzet-bachelorproef}

%% TODO: Het is gebruikelijk aan het einde van de inleiding een overzicht te
%% geven van de opbouw van de rest van de tekst. Deze sectie bevat al een aanzet
%% die je kan aanvullen/aanpassen in functie van je eigen tekst.

De rest van deze bachelorproef is als volgt opgebouwd:

In Hoofdstuk~\ref{ch:methodologie} wordt de methodologie toegelicht en worden de gebruikte onderzoekstechnieken besproken om een antwoord te kunnen formuleren op de onderzoeksvragen.

%% TODO: Vul hier aan voor je eigen hoofstukken, één of twee zinnen per hoofdstuk

In Hoofdstuk~\ref{ch:conclusie}, tenslotte, wordt de conclusie gegeven en een antwoord geformuleerd op de onderzoeksvragen. Daarbij wordt ook een aanzet gegeven voor toekomstig onderzoek binnen dit domein.

