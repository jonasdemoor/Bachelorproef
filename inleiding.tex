%%=============================================================================
%% Inleiding
%%=============================================================================

\chapter{Inleiding}
\label{ch:inleiding}

%De inleiding moet de lezer alle nodige informatie verschaffen om het onderwerp te begrijpen zonder nog externe werken te moeten raadplegen \autocite{Pollefliet2011}. Dit is een doorlopende tekst die gebaseerd is op al wat je over het onderwerp gelezen hebt (literatuuronderzoek).

%Je verwijst bij elke bewering die je doet, vakterm die je introduceert, enz. naar je bronnen. In \LaTeX{} kan dat met het commando \texttt{$\backslash${textcite\{\}}} of \texttt{$\backslash${autocite\{\}}}. Als argument van het commando geef je de ``sleutel'' van een ``record'' in een bibliografische databank in het Bib\TeX{}-formaat (een tekstbestand). Als je expliciet naar de auteur verwijst in de zin, gebruik je \texttt{$\backslash${}textcite\{\}}.
%Soms wil je de auteur niet expliciet vernoemen, dan gebruik je \texttt{$\backslash${}autocite\{\}}. Hieronder een voorbeeld van elk.

%\textcite{Knuth1998} schreef een van de standaardwerken over sorteer- en zoekalgoritmen. Experten zijn het erover eens dat cloud computing een interessante opportuniteit vormen, zowel voor gebruikers als voor dienstverleners op vlak van informatietechnologie~\autocite{Creeger2009}.

%\section{Stand van zaken}
%\label{sec:stand-van-zaken}

%% TODO: deze sectie (die je kan opsplitsen in verschillende secties) bevat je
%% literatuurstudie. Vergeet niet telkens je bronnen te vermelden!

%\lipsum[7-20]

In dit hoofdstuk wordt er een korte inleiding gegeven over de basisprincipes van RAID, aangezien RAID-Z een softwarematige vorm van RAID is. Daarnaast wordt de geschiedenis en globale werking van ZFS reeds kort besproken. Aan het einde van dit hoofdstuk kunnen de probleemstelling en onderzoeksvragen worden teruggevonden, samen met de verdere indeling van deze bachelorproef.

\section{RAID}

Al van oudscher worden magnetische harde schijven gebruikt als opslagmedium voor data. Maar reeds in de jaren 80 zagen onderzoekers in dat I/O-performance een bottleneck zou vormen voor computersystemen in de toekomst. Terwijl geheugenchips en processoren steeds sneller werden, bevonden opslagmedia zich in een impasse \autocite{DavidA.Paterson1987}.

In de paper \textit{"A Case for Redundant Arrays of Inexpensive Disks"}  formuleerden \textcite{DavidA.Paterson1987} en zijn collega's voor het eerst de term 'RAID', wat een acroniem is voor 'Redundant Arrays of Inexpensive Disks'. De oorspronkelijke idee achter RAID was dat een verzameling van goedkopere schijven performanter zou zijn dan grotere en duurdere mainframeschijven van die tijd. 

Naast performantie en kost was betrouwbaarheid (reliability)  ook een belangrijke factor. Als bv. één of meerdere schijven van de array falen, dan mag dit geen invloed hebben op de werking van de rest van de verzameling schijven. Daarom introduceerden de onderzoekers de zgn. "RAID levels"    \autocite{DavidA.Paterson1987}, die vandaag de dag nog steeds in gebruik zijn. Er bestaan een aantal RAID-niveaus, waarvan de voornaamste zullen besproken worden.

Bij het bouwen van RAID-systemen worden er gebruikelijk drie aspecten in beschouwing genomen: \textbf{performantie} (performance), \textbf{betrouwbaarheid} (reliability) en \textbf{capaciteit} (capacity). Een RAID-niveau is in principe niets anders dan een balans tussen deze verschillende eigenschappen; meestal zullen er dus één of meerdere trade-offs moeten gemaakt worden \autocite{OSThreePiecesRemzi2015}.

\section{Probleemstelling en Onderzoeksvragen}
\label{sec:onderzoeksvragen}

%% TODO:
%% Uit je probleemstelling moet duidelijk zijn dat je onderzoek een meerwaarde
%% heeft voor een concrete doelgroep (bv. een bedrijf).
%%
%% Wees zo concreet mogelijk bij het formuleren van je
%% onderzoeksvra(a)g(en). Een onderzoeksvraag is trouwens iets waar nog
%% niemand op dit moment een antwoord heeft (voor zover je kan nagaan).

\section{Opzet van deze bachelorproef}
\label{sec:opzet-bachelorproef}

%% TODO: Het is gebruikelijk aan het einde van de inleiding een overzicht te
%% geven van de opbouw van de rest van de tekst. Deze sectie bevat al een aanzet
%% die je kan aanvullen/aanpassen in functie van je eigen tekst.

De rest van deze bachelorproef is als volgt opgebouwd:

In Hoofdstuk~\ref{ch:methodologie} wordt de methodologie toegelicht en worden de gebruikte onderzoekstechnieken besproken om een antwoord te kunnen formuleren op de onderzoeksvragen.

%% TODO: Vul hier aan voor je eigen hoofstukken, één of twee zinnen per hoofdstuk

In Hoofdstuk~\ref{ch:conclusie}, tenslotte, wordt de conclusie gegeven en een antwoord geformuleerd op de onderzoeksvragen. Daarbij wordt ook een aanzet gegeven voor toekomstig onderzoek binnen dit domein.

