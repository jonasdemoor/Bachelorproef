%%=============================================================================
%% Inleiding
%%=============================================================================

\chapter{Inleiding}
\label{ch:inleiding}

\section{Probleemstelling en Onderzoeksvragen}
\label{sec:onderzoeksvragen}

%% TODO:
%% Uit je probleemstelling moet duidelijk zijn dat je onderzoek een meerwaarde
%% heeft voor een concrete doelgroep (bv. een bedrijf).
%%
%% Wees zo concreet mogelijk bij het formuleren van je
%% onderzoeksvra(a)g(en). Een onderzoeksvraag is trouwens iets waar nog
%% niemand op dit moment een antwoord heeft (voor zover je kan nagaan).

Aangezien (een deel van) de RAID-functionaliteit van BTRFS nog niet als \textit{production-ready} wordt beschouwd \autocite{Project2017}, zal ZFS worden beschouwd en geëvalueerd als volwaardig alternatief voor klassieke RAID-oplossingen op Linux-systemen. 

Deze bachelorproef zal een antwoord vinden op volgende vragen:

\begin{itemize}
	\item{Wat zijn de grootste verschillen tussen een klassieke RAID-oplossing en ZFS RAID-Z?}
	\item{Hoe is de architectuur van ZFS opgebouwd en op welke manieren tracht het oplossingen te vinden voor de problemen die zich voordoen bij andere bestandssystemen en RAID-opstellingen?}
  \item{Hoe staat het met data-integriteit en \gls{performantie}\footnote{Met 'performantie' wordt het aantal I/O's per seconde en de globale schijf- en CPU-belasting bedoeld.} bij ZFS onder verschillende workloads en toepassingen?}
\end{itemize}

\section{Opzet van deze bachelorproef}
\label{sec:opzet-bachelorproef}

%% TODO: Het is gebruikelijk aan het einde van de inleiding een overzicht te
%% geven van de opbouw van de rest van de tekst. Deze sectie bevat al een aanzet
%% die je kan aanvullen/aanpassen in functie van je eigen tekst.

De rest van deze bachelorproef is als volgt opgebouwd:

In Hoofdstuk~\ref{ch:methodologie} wordt de methodologie toegelicht en worden de gebruikte onderzoekstechnieken besproken om een antwoord te formuleren op de onderzoeksvragen.

In Hoofdstuk~\ref{ch:h2} wordt er een korte inleiding gegeven op de geschiedenis en de algemene werking van RAID-systemen. Ook ZFS en RAID-Z worden reeds kort toegelicht.

In Hoofdstuk~\ref{ch:h3} wordt er een globaal overzicht gegeven van de architectuur en ontwerpprincipes van ZFS. In de daaropvolgende hoofdstukken worden de belangrijkste onderdelen en functionaliteiten wat meer uitgediept. 

In Hoofdstuk \ref{ch:h4} wordt het opslagmodel van het ZFS-bestandssysteem besproken. Onder andere de datastructuur en het transactiemodel van ZFS komen aan bod.

In Hoofdstuk~\ref{ch:conclusie}, tenslotte, wordt de conclusie gegeven en een antwoord geformuleerd op de onderzoeksvragen. Daarbij wordt ook een aanzet gegeven voor toekomstig onderzoek binnen dit domein.

