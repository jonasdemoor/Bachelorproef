%%=============================================================================
%% Samenvatting
%%=============================================================================

%% TODO: De "abstract" of samenvatting is een kernachtige (~ 1 blz. voor een
%% thesis) synthese van het document.
%%
%% Deze aspecten moeten zeker aan bod komen:
%% - Context: waarom is dit werk belangrijk?
%% - Nood: waarom moest dit onderzocht worden?
%% - Taak: wat heb je precies gedaan?
%% - Object: wat staat in dit document geschreven?
%% - Resultaat: wat was het resultaat?
%% - Conclusie: wat is/zijn de belangrijkste conclusie(s)?
%% - Perspectief: blijven er nog vragen open die in de toekomst nog kunnen
%%    onderzocht worden? Wat is een mogelijk vervolg voor jouw onderzoek?
%%
%% LET OP! Een samenvatting is GEEN voorwoord!

%%---------- Nederlandse samenvatting -----------------------------------------
%%
%% TODO: Als je je bachelorproef in het Engels schrijft, moet je eerst een
%% Nederlandse samenvatting invoegen. Haal daarvoor onderstaande code uit
%% commentaar.
%% Wie zijn bachelorproef in het Nederlands schrijft, kan dit negeren en heel
%% deze sectie verwijderen.

%\IfLanguageName{english}{%
%\selectlanguage{dutch}
%\chapter*{Samenvatting}
%\lipsum[1-4]
%\selectlanguage{english}
%}{}

%%---------- Samenvatting -----------------------------------------------------
%%
%% De samenvatting in de hoofdtaal van het document

\chapter*{\IfLanguageName{dutch}{Samenvatting}{Abstract}}

%\lipsum[1-4]

RAID (Redundant Array of Independent Disks) is een technologie die al lange tijd is ingeburgerd in bedrijven. Systeembeheerders gebruiken RAID vooral om ervoor te zorgen dat één of meerdere defecte schijven niet kunnen leiden tot dataverlies, maar dit hoeft niet noodzakelijk het geval te zijn. 

In het laatste decennium zijn softwaregebaseerde RAID-oplossingen steeds populairder geworden. Eén van deze oplossingen is RAID-Z, een softwarematige RAID die deel uitmaakt van de ZFS storage stack. ZFS is een geavanceerd bestandssysteem ontwikkeld door het vroegere Sun Microsystems in het begin van de jaren 2000. 

In deze bachelorproef wordt achterhaald of ZFS en RAID-Z een goed alternatief zouden vormen voor een meer traditionele RAID-oplossingen, zoals een hardwaregebaseerde RAID. De motivatie voor het voeren van een onderzoek naar een alternatieven voor RAID is hoofdzakelijk omdat traditionele RAID5 arrays nog steeds onderhevig zijn aan het zgn. "RAID5 write hole", waarbij dataverlies kan optreden bij bijvoorbeeld een stroompanne. Ook bieden de meeste RAID-controllers geen bescherming tegen silent data corruption.

De scriptie is opgedeeld in twee grote delen: een theoretisch deel en een praktisch deel. In het theoretische deel worden de basisprincipes van RAID en de architectuur van ZFS in vogelvlucht overlopen. Het praktische gedeelte behandelt voornamelijk de performantie en betrouwbaarheid van ZFS en RAID-Z. 
De performantietesten en het grootste deel van het praktische deel werd uitgevoerd m.b.v. een HP desktopsysteem; de betrouwbaarheidstesten werden uitgevoerd met een virtuele machine via VirtualBox. Hierbij werd geconstateerd dat ZFS zijn beloftes m.b.t. betrouwbaarheid waarmaakt en de data goed beschermt tegen datacorruptie en hardwarefalen. Performantie van ZFS is uitstekend te noemen; de performantie werd vergeleken met die van Linux MD, een softwarematige RAID voor Linux, en werd getest m.b.v. de Phoronix Test Suite.

Afhankelijk van de use case (situatie, nodige opslagcapaciteit, beschikbare hardware) vormt ZFS met RAID-Z in het merendeel van de gevallen een uitstekend alternatief voor een klassieke RAID-oplossing. In de toekomst zullen ZFS en andere COW-bestandssystemen, zoals APFS, ReFS en BTRFS, naar alle waarschijnlijkheid interessanter worden voor dagelijks gebruik; een vergelijkende studie tussen deze verschillende bestandssystemen zou nog een interessante toevoeging zijn aan deze scriptie.
