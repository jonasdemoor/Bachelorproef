
%%=============================================================================
%% H7 - ZFS Datasets
%%=============================================================================

\chapter{ZFS Datasets}
\label{ch:h7}

In dit hoofdstuk worden ZFS datasets of bestandssystemen besproken. Er wordt onder andere getoond hoe men bestandssystemen kan aanmaken, wijzigen en delen met andere computers op het netwerk. 

\section{Verschillen met traditionele bestandssystemen}

In essentie verschillen datasets (of bestandssystemen) binnen ZFS niet zo heel veel van andere, "traditionele" bestandssystemen. Een dataset is in principe een verzameling van data dat een bepaalde naam heeft; deze dataset vormt een logische eenheid van beheer. Dit komt grotendeels overeen met partities bij andere bestandssystemen, waarbij een (deel van) een schijf wordt gereserveerd voor een bepaald type gebruik (zoals gebruikersmappen) \autocite{Lucas2015}.

Toch zijn er een aantal grote verschillen, zoals bijvoorbeeld de manier waarop er omgegaan wordt met beschikbare schijfruimte.  

\subsection{Gebruik van de beschikbare opslagcapaciteit}

Een groot verschil tussen bestandssystemen bij ZFS en vele andere bestandssystemen, is dat ZFS datasets gebruik maken van de voordelen van storage pools. ZFS legt bijna geen limieten op aan de grootte van bestandssystemen; de enige limiet die aan de grootte van bestandssystemen wordt opgelegd, is de grootte van de storage pool. Dit is een groot verschil met traditionele bestandssystemen: alvorens een partitie aan te maken en te formatteren, moet de systeembeheerder eerst nadenken over de grootte en lay-out van deze partitie en van de rest van de schijf of schijven. Na deze beslissingen genomen te hebben, maakt de gebruiker deze partitie aan. \autocite{Lucas2015}.

Standaard neemt een ZFS dataset de ruimte van een pool in die het nodig heeft; de systeembeheerder moet zich dus niet bezighouden met het vooraf instellen van de grootte van een dataset. Het is echter wel mogelijk om de groei van een dataset tegen te gaan d.m.v. quota's en reservaties \autocite{FBSDDP2017}; het gebruik van ZFS properties op datasets komt later in dit hoofdstuk nog aan bod. 

\subsection{Verschillende types van datasets}

Een dataset kan binnen ZFS een aantal vormen aannemen. Eén van de meest voorkomende vormen van een dataset is een \textbf{file system} of bestandssysteem: dit is het equivalent van een klassiek bestandssysteem, met bestanden, mappen, permissies enzovoorts. Daarnaast bestaan er nog andere datasettypes, zoals snapshots, clones, volumes (of zvols) en bookmarks \autocite{Lucas2015}. Deze verschillende types zullen iets uitgebreider worden besproken in de volgende sectie.  

\subsection{Hiërarchische structuur van ZFS datasets}

Datasets worden - net zoals zpools - binnen ZFS voorgesteld als objecten, met elk bepaalde eigenschappen (of properties). Traditionele bestandssystemen hebben ook bepaalde eigenschappen die kunnen veranderd worden, maar ZFS gaat hier nog een stap verder in. Net zoals bij VDEV's worden datasets voorgesteld in een boomstructuur: elke dataset heeft een ouder en eventueel kinderen. Dit maakt het mogelijk om overerving toe te passen op datasets; hierbij erven de kinderen van een bepaalde dataset de properties over van hun ouder \autocite{FBSDDP2017}.

Datasets maken het voor de systeembeheerder makkelijk om verschillende soorten data van elkaar te scheiden; op deze datasets kunnen dan bepaalde properties worden ingesteld m.b.t. bijvoorbeeld gebruikersrechten. Met behulp van ouder-kind relaties tussen datasets kunnen zowel algemene eigenschappen als meer specifieke eigenschappen worden ingesteld. Er kan bijvoorbeeld een dataset worden aangemaakt voor een project; deze beschikt dan over algemene eigenschappen, zoals toegangsrechten voor de verschillende teams die aan het project werken. Binnen deze ouder-dataset kunnen kind-datasets worden aangemaakt met meer specifieke eigenschapppen; elke dataset behoort bijvoorbeeld toe aan een specifiek team en heeft dan ook eigenschappen die uniek zijn voor dat bepaald team, bovenop de reeds overgeërfde eigenschappen van de ouder-dataset (voorbeeld naar \textcite{Lucas2015}).

\section{Aanmaken \& beheren van ZFS datasets}

Voor het beheer van datasets wordt het commando \texttt{zfs} gebruikt; de syntax komt grotendeels overeen met die van \texttt{zpool}, het commando om ZFS pools te beheren. 

Als vertrekpunt wordt er een zpool gebruikt bestaande uit één RAID-Z1 VDEV en drie fysieke VDEV's:

\begin{lstlisting}[language=bash,style=command_style]
  $ zpool create storage raidz1 /dev/sda /dev/sdb /dev/sdc
  $ zpool status
    pool: storage
   state: ONLINE
    scan: none requested
  config:

	NAME        STATE     READ WRITE CKSUM
	storage     ONLINE       0     0     0
	  raidz1-0  ONLINE       0     0     0
	    sda     ONLINE       0     0     0
	    sdb     ONLINE       0     0     0
	    sdc     ONLINE       0     0     0

  errors: No known data errors
\end{lstlisting}

\subsection{Bekijken en aanmaken van datasets}

Om de huidige aanwezige datasets te bekijken, gebruik je het commando \texttt{zfs list}:

\begin{lstlisting}[language=bash,style=command_style] 
  $ zfs list
  NAME      USED  AVAIL  REFER  MOUNTPOINT
  storage   464K   898G   128K  /storage
\end{lstlisting}

In dit geval is er reeds automatisch een dataset aangemaakt met de naam 'storage' en als mountpoint '/storage'. Deze heeft dezelde naam als de zpool: deze is dus m.a.w. de parent dataset van alle andere datasets die binnen deze pool zullen worden aangemaakt.

Er kan ook een lijst worden opgevraagd van alle aangekoppelde (\textit{mounted}) bestandssystemen m.b.v. het UNIX-commando \texttt{df}:

\begin{lstlisting}[language=bash,style=command_style] 
  $ df -h
  Filesystem               Size  Used Avail Use% Mounted on
  devtmpfs                 4.9G     0  4.9G   0% /dev
  tmpfs                    4.9G     0  4.9G   0% /dev/shm
  tmpfs                    4.9G  920K  4.9G   1% /run
  tmpfs                    4.9G     0  4.9G   0% /sys/fs/cgroup
  /dev/mapper/fedora-root   15G  1.8G   14G  12% /
  tmpfs                    4.9G  4.0K  4.9G   1% /tmp
  /dev/sdd1                976M  138M  772M  16% /boot
  tmpfs                    993M     0  993M   0% /run/user/1000
  storage                  899G  128K  899G   1% /storage 
\end{lstlisting}

Zoals kan worden afgeleid uit de uitvoer van bovenstaand commando, is de ZFS dataset inderdaad gekoppeld aan de map \texttt{/storage}.

Om binnen de dataset \texttt{storage} een nieuwe dataset aan te maken, gebruik je het commando \texttt{zfs create}. Met behulp van dit commando kunnen er verschillende soorten ZFS datasets worden aangemaakt. Hieronder volgt een overzicht van de verschillende soorten datasets en hoe deze kunnen worden ingezet.

\subsubsection{Filesystems}

Filesystems (of bestandssystemen) zijn de meest voorkomende datasettypes. Qua eigenschappen en werking komen ze overeen met traditionele bestandssystemen: beide dienen om bestanden en mappen op te slaan, bevatten een POSIX-achtig permissiemodel en inodes \autocite{Lucas2015}. 

Om het voorgaande voorbeeld van de verschillende teams met verschillende projecten wat te illustreren, maken we onder de root dataset \texttt{/storage} de dataset \texttt{/storage/projects} aan. Daarna worden de andere datasets aangemaakt onder de parent \texttt{/storage/projects}.

\begin{lstlisting}[language=bash,style=command_style] 
  $ zfs create storage/projects
  $ zfs create storage/projects/www
  $ zfs create storage/projects/dev
  $ zfs list
  NAME                   USED  AVAIL  REFER  MOUNTPOINT
  storage                767K   898G   133K  /storage
  storage/projects       394K   898G   139K  /storage/projects
  storage/projects/dev   128K   898G   128K  /storage/projects/dev
  storage/projects/www   128K   898G   128K  /storage/projects/www
\end{lstlisting}

Uit de uitvoer van \texttt{zfs list} kan men de makkelijk de ouder-kindrelaties opmaken: \texttt{storage} is de ouder van projects en projects is op zijn beurt opnieuw ouder van \texttt{dev} en \texttt{www}.

\subsubsection{Volumes}

ZFS volumes (of zvols) zijn 
