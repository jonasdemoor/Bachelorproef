%%=============================================================================
%% Conclusie
%%=============================================================================

\chapter{Conclusie}
\label{ch:conclusie}

%% TODO: Trek een duidelijke conclusie, in de vorm van een antwoord op de
%% onderzoeksvra(a)g(en). Wat was jouw bijdrage aan het onderzoeksdomein en
%% hoe biedt dit meerwaarde aan het vakgebied/doelgroep? Reflecteer kritisch
%% over het resultaat. Had je deze uitkomst verwacht? Zijn er zaken die nog
%% niet duidelijk zijn? Heeft het ondezoek geleid tot nieuwe vragen die
%% uitnodigen tot verder onderzoek?

%\lipsum[76-80]

Doorheen deze bachelorproef is het (hopelijk) duidelijk geworden dat het gebruik van de term 'bestandssysteem' ZFS eigenlijk een beetje onrecht aandoet. De afkorting 'ZFS' mag dan wel Zettabyte Filesystem betekenen, toch is ZFS veel meer dan een bestandssysteem alleen: het is een volledig nieuwe implementatie van de traditionele opslagstack zoals we die al jaren gewend zijn. Daar waar meer traditionele oplossingen vaak bestaan uit verschillende, losse onderdelen die met elkaar samenwerken, bestaat de ZFS storage stack uit een aantal nauw samenwerkende lagen. Toch voelt dit niet aan als één groot, log geheel; functionaliteiten zijn duidelijk afgebakend en dit weerspiegelt zich ook in het gebruik en beheer ervan.   

In Hoofdstuk \ref{ch:h9} werden er enkele testen uitgevoerd om het gedrag van ZFS te toetsen tijdens ongunstige omstandigheden. In deze hypothetische testen werd een RAID-Z-opstelling blootgesteld aan twee obstakels: het falen van een harde schijf en het volschrijven van een VDEV met random data. ZFS slaagde er perfect in om datacorruptie en verlies van data te voorkomen. 

%Een traditionele RAID-opstelling zou ook slagen in de eerstgenoemde test, maar zou falen bij de tweede; dit is te wijten aan het feit dat een traditionele RAID-controller geen geavanceerd checksummechanisme op blokniveau heeft. Voor een RAID-controller zijn enkel blokken van belang, of deze nu corrupt zijn of niet maakt voor de controller geen verschil. Dit is meteen ook een groot voordeel dat ZFS heeft t.o.v. traditionele bestandssystemen en RAID-controllers.

Naast checksumming heeft ZFS ook tal van andere voordelen, zoals snapshotting van data, flexibele storage pools en ZFS volumes. Bovenop deze zvols kunnen andere bestandssystemen worden aangemaakt; deze profiteren dan onmiddellijk van de mogelijkheden die ZFS te bieden heeft, waaronder compressie en Copy-On-Write. Met behulp van snapshots kan er snel data worden hersteld bij bijvoorbeeld een gebruikersfout. Toch zijn enkel snapshots niet voldoende als back-up: als er meerdere schijven van bijvoorbeeld een RAID-Z array falen, dan zijn de gemaakte snapshots ook verloren. Daarom blijft een goede back-upstrategie onontbeerlijk.

Daarnaast bleek ZFS toch niet zo flexibel te zijn als eerst gedacht. Storage pools zijn makkelijk uit te breiden door schijven toe te voegen, maar de VDEV's binnen een pool zijn dat meestal niet. Enkel stripes en mirrors kunnen na creatie uitgebreid worden met schijven; in andere gevallen moet er een extra VDEV worden aangemaakt. Daarom is ZFS meestal geen goede oplossing voor bijvoorbeeld start-ups die klein willen beginnen en later hun opslagcapaciteit nog willen uitbreiden. In zo'n situaties moet eerst de bestaande data in de pool worden geback-upped; vervolgens moet de bestaande pool worden verwijderd. Uiteindelijk kan men dan een nieuwe pool aanmaken met nieuwe VDEV's en kan de data worden teruggezet.

Uit de performantietesten in Hoofdstuk \ref{ch:h8} bleek dat ZFS het goed deed tegen Linux MD, een softwarematige RAID op Linux. Deze prestaties zijn grotendeels te danken aan het uitstekende cachingmechanisme binnen ZFS. Ook bleek ZFS goed te functioneren op een relatief oud systeem (+/- 6 jaar) met al wat oudere hardware. Echter kan het gebrek aan een relatief krachtige processor een obstakel vormen bij het gebruik van ZFS op bijvoorbeeld NAS-systemen, die meestal uitgerust zijn met een zuinige maar ook minder krachtige CPU. Checksumming is uit te schakelen, maar dan worden ook de voordelen van ZFS teniet gedaan. In zo'n gevallen kan men beter overschakelen op een klassieke en (eventueel softwarematige) RAID-oplossing.

ZFS is verre van perfect, en Copy-On-Write is zeker geen nieuw concept. Toch bleek uit deze bachelorproef dat Sun's nu meer dan tien jaar oude bestandssysteem de tand des tijds goed heeft doorstaan: in de laatste jaren zien we dat bestandssystemen met gelijkaardige functionaliteiten steeds meer de kop beginnen opsteken. HAMMER (DragonFlyBSD) \autocite{DragonFlyBSD2017}, BTRFS (Linux) \autocite{BTRFSProject2017}, ReFS (Microsoft Windows) \autocite{Microsoft2016} en het recent aangekondigde APFS (Apple) \autocite{Apple2017} zijn voorbeelden van bestandssystemen die qua werking en features hoofdzakelijk overeenkomen met ZFS.


