%%=============================================================================
%% Conclusie
%%=============================================================================

\chapter{Conclusie}
\label{ch:conclusie}

%% TODO: Trek een duidelijke conclusie, in de vorm van een antwoord op de
%% onderzoeksvra(a)g(en). Wat was jouw bijdrage aan het onderzoeksdomein en
%% hoe biedt dit meerwaarde aan het vakgebied/doelgroep? Reflecteer kritisch
%% over het resultaat. Had je deze uitkomst verwacht? Zijn er zaken die nog
%% niet duidelijk zijn? Heeft het ondezoek geleid tot nieuwe vragen die
%% uitnodigen tot verder onderzoek?

%\lipsum[76-80]

Doorheen deze bachelorproef is het (hopelijk) duidelijk geworden dat het gebruik van de term 'bestandssysteem' ZFS eigenlijk een beetje onrecht aandoet. De afkorting 'ZFS' mag dan wel Zettabyte Filesystem betekenen, toch is ZFS veel meer dan een bestandssysteem alleen: het is een volledig nieuwe implementatie van de traditionele opslagstack zoals we die al jaren gewend zijn. Daar waar meer traditionele oplossingen vaak bestaan uit verschillende, losse onderdelen die met elkaar samenwerken, bestaat de ZFS storage stack uit een aantal nauw samenwerkende lagen. Toch voelt dit niet aan als een groot, log geheel; functionaliteiten zijn duidelijk afgebakend en dit weerspiegelt zich ook in het gebruik en beheer.   

Toen ZFS voor het eerst op het toneel verscheen, waren vele techneuten laaiend enthousiast: problemen zoals datacorruptie en inflexibele opslag zouden eindelijk van de baan zijn.  
